\section{Lower Power Research Papers}

\subsection{Mobile AP Thermal Modeling}

\subsubsection*{Thermal Issues in Mobile AP}
\begin{itemize}
    \item 온도 문제는 시간이 흐르면서 더욱 심각해지고 있음
    \item Cooling System (Heat sink, 쿨링 팬, 수냉 등): 공간 부족으로 인해 부적합
    \item DVFS 등의 DTM: 성능 저하 초래 문제
    \item 온도 문제는 early-design stage부터 고려되어야 함
    \begin{itemize}
        \item Proper thermal model이 반드시 필요함
        \item Chip의 design 정보와 power trace에 근거하여 모델링하였음
    \end{itemize}
\end{itemize}

\subsubsection*{Thermal Modeling Methodology}
\begin{itemize}
    \item Target Mobile AP
    \begin{itemize}
        \item PoP(Package-on-Package)-type mobile AP
        \item Processor die: 4개의 big core + 4개의 small core, TSMC 22nm 공정
    \end{itemize}
    \item 프로세서를 덮고 있는 메모리는 차가움
    \begin{itemize}
        \item 메모리와 package를 떼어내고 투명한 crystal package를 씌운 뒤 IR 카메라로 온도 측정하여 모델 구성
        \item 원본 chip을 이용하여 validation 수행
    \end{itemize}
    \item Processor가 더 뜨거운데 아래에 위치하는 이유
    \begin{itemize}
        \item 위쪽에 있으면 power 공급이 불안정함 (프로세서에는 치명적)
        \item DRAM도 온도가 증가하면 refresh rate가 증가함
        \item 정해진 위치는 없으며, 위에 배치하기도 함
    \end{itemize}
    \item Model Design Tool: HotSpot (University of Virginia)
    \item 제조사로부터 제공된 floorplan, material properties를 이용, 측정한 subsystem별 power trace를 이용
\end{itemize}
\begin{figures}
    \fig{pop-type-mobile-ap.png}{.8}
\end{figures}

\subsubsection*{Thermal Modeling Results}
\begin{itemize}
    \item Big CPU, Little CPU, MEMC0 등에서는 MAE $\approx$ 2.34 - 3.51$\cel$로 거의 정확
    \item GPU, CCI 등에서는 MAE $\approx$ 4.55 - 6.05$\cel$로 다소 부정확
    \begin{itemize}
        \item 크기가 큰 subsystem은 부분별로 온도가 다를 수 있음 $\Rightarrow$ subsystem별로 하나의 온도만 측정되어 부정확
        \item Big, thermally critical subsystem을 여러 sublock으로 나누고, 각 core의 $\mathrm{util_{CPU}}$를 통해 소모 전력 예측
        \item Selective fine-grained modeling
    \end{itemize}
    \item GPU, CCI의 MAE $\approx$ 2.08 - 2.56$\cel$로 다소 정확해짐
\end{itemize}

\subsubsection*{Validation}
\begin{itemize}
    \item Real mobile application을 통한 검증
    \begin{itemize}
        \item Adobe reader, File compressor, MX Player 등
        \item 평균 MAE $\approx$ 1.58$\cel$: 정확
    \end{itemize}
    \item Thermal sensor를 통한 검증
    \begin{itemize}
        \item Modeling 과정과는 달리 package와 메모리를 얹은 채로 진행
        \item 제조사 제공 floorplan을 통해 열 센서의 위치를 알고있는 상태에서 측정
        \item 평균 MAE $\approx$ 1.75$\cel$: 정확
    \end{itemize}
\end{itemize}

\subsubsection*{Summary}
\begin{itemize}
    \item Mobile AP의 온도 map을 관찰한 첫 연구
    \item 제시된 model이 온도를 정확하게 예측함
\end{itemize}

\subsection{M-DTM: Migration-based Dynamic Themral Management}

\subsubsection*{Issues of Conventional DTM}
\begin{itemize}
    \item Small core를 사용하지 않음
    \begin{itemize}
        \item Thermal emergency를 빠르게 해결하지 못함
        \item 앱이 big core의 가장 높은 $f$로 충분히 돌 시간이 없음
    \end{itemize}
    \item Small core를 사용하여 big core를 냉각시키는 아이디어
\end{itemize}

\subsubsection*{Proposed Scheme}
\begin{itemize}
    \item $T_\mathrm{big}>T_\mathrm{th}$: Task scheduler가 모든 app을 small core로 migrate
    \item $T_\mathrm{big}\leq T_\mathrm{th}$: Task scheduler가 모든 app을 big core로 migrate
    \item Scaling governor은 big, small core에서 가장 높은 $f$로 돌게 함 $\Rightarrow$ 성능 저하 다소 cover
    \item Potential Problems: Migration Overhead
    \begin{itemize}
        \item 과거에는 migration overhead가 다소 컸음 (1-5초) $\Rightarrow$ 최근에는 거의 없음
        \item Big-little migration
        \begin{itemize}
            \item Live migration: 수행하던 부분만 먼저 migration하는 방법
            \item Shared LLC 효과: 현재 수행하고 있는 부분이 DRAM이 아닌 cache에 들어가 있어 훨씬 빠름
        \end{itemize}
        \item Big-big migration: 거의 작음 (Performace overhead $\approx$ 0.4\%)
    \end{itemize}
\end{itemize}

\subsubsection*{Experimental Results}
\begin{itemize}
    \item 성능
    \begin{itemize}
        \item 실행 시간: Conv. DTM에 비해 10.6\% 낮음
        \item Highest $f$로 동작하는 시간: Conv. DTM에 비해 2.5배 높음
        \item Small core는 pipeline stage가 적고 cache가 작음 $\Rightarrow$ branch misprediction 많고 L2 cache hit이 높을수록 유리
    \end{itemize}
    \newpage
    \item 온도
    \begin{itemize}
        \item 평균 온도: 거의 유사
        \item 최고 온도: Conv. DTM에 비해 7.4$\cel$ 낮음
    \end{itemize}
    \item 에너지 소모: Conv. DTM에 비해 3.6\% 낮음
    \begin{itemize}
        \item 전력은 Conv. DTM보다 더 소모하나 실행 시간이 대폭 줄어들어 에너지 소모는 감소
    \end{itemize}
\end{itemize}

\subsubsection*{Migration among Big Cores + M-DTM/DVFS}
\begin{itemize}
    \item IPA는 M-DTM보다 효율적인 방법. 그러나 IPA는 예측이 어렵고 부정확하다는 단점이 있음
    \item Big core간의 migration을 먼저 수행하고, 모든 big core의 온도가 높으면 M-DTM이나 DVFS 수행
\end{itemize}

\subsubsection*{DTM, Ideal IPA, Advanced M-DTM 성능 비교}
\begin{itemize}
    \item 실행 시간
    \begin{itemize}
        \item 모든 경우에 대해, Advanced M-DTM $>$ IPA $>$ DTM
        \item App의 수가 많아지면 M-DTM, Advanced M-DTM의 효과는 저하
    \end{itemize}
    \item System-wide EDP (Energy-delay product): M-DTM $\gg$ IPA $\geq$ Advanced M-DTM
    \item DTM 발생 시간 비율 (= 1 - highest $f$ 작동 시간)
    \begin{itemize}
        \item IPA $>$ M-DTM $>$ Advanced M-DTM
        \item $T_\mathrm{th}$가 높고 app 수가 적으면 migration만으로도 DTM을 거의 없앨 수 있음
    \end{itemize}
\end{itemize}

\subsubsection*{Prime core}
\begin{itemize}
    \item Prime core는 big core보다도 성능이 좋은 core
    \item 장점: 성능 향상 / 단점: 면적 overhead, prime core에서의 thermal emergency
    \item Heatpipe: 액체가 뜨거운 core에서 증발, 차가운 core에서 응결하여 순환하는 냉각 방식
\end{itemize}
\begin{figures}
    \fig{heatpipe.png}{0.4}
\end{figures}

\subsubsection*{Future DTM Researches}
\begin{itemize}
    \item Thermal Control
    \begin{itemize}
        \item Ambient 온도: Ambient 온도로 인해 같은 환경에서는 성능이 flip될 수 있음
        \item Ambient 온도, secondary heat path 등으로 인한 side effect 고려해야 함
        \item Instruction별로 다른 복잡도를 고려해야 함
    \end{itemize}
    \item Hardware PID Control
    \begin{itemize}
        \item Hardware 단위로 $V$/$f$ 조절 $\Rightarrow$ 더 빠르게 대응할 수 있음 (fine-grain thermal control)
        \item 예: on-chip voltage regulator
        \item Gain이 어느 정도일지는 아직 미지수
    \end{itemize}
    \item Task Migration
    \item Overclocking: 하나의 core만 사용할 때는 아주 짧은 시간동안 overclocking
\end{itemize}

\subsection{Ambient Temperature Aware VM Allocation}

\subsubsection*{Motivation: Data Center}
\begin{itemize}
    \item 데이터 센터의 서버는 서버의 특징, workload의 특징에 따라 온도가 다름, 대류로 인해 열이 주변으로 전달됨
    \item 데이터 센터는 주로 heterogeneous 서버로 구성되어 있음
    \item Ambient 온도에 의해 성능 저하가 발생할 수 있음 $\Rightarrow$ 이를 고려하여 성능 저하 최소화
\end{itemize}

\subsubsection*{Related Researches}
\begin{itemize}
    \item Consolidation: 서버의 성능에 따라 VM을 모아서 배치하는 방법
    \begin{itemize}
        \item 서버의 누설 전력이 클 때 사용
        \item 장점: 에너지 소모 감소
        \item 단점: 잦은 throttling, 자원 경합, ambient 요소 고려하지 못함
    \end{itemize}
    \item Load balancing: 동적 수행 능력을 고려하여 서버에 VM을 분산 배치하는 방법
    \begin{itemize}
        \item 서버의 누설 전력이 작을 때 사용
        \item 장점: 에너지 소모 감소, 성능 증가
        \item 단점: ambient 요소 고려하지 못함
    \end{itemize}
\end{itemize}

\subsubsection*{Overall Scheme}
\begin{itemize}
    \item Ambient-aware dynamic computing capability estimation model
    \begin{itemize}
        \item 주변 온도, $f_\mathrm{CPU}$, LLC 크기, $\mathrm{util_{CPU}}$, $\mathrm{util_{mem}}$ 고려 $\Rightarrow$ 선형 회귀법으로 모델 구성
        \item 모델을 이용하여 (ambient-aware dynamic computing) capability 계산
    \end{itemize}
    \item LLC의 MPKI(Misses Per Kilo-Instruction)를 고려하여 VM의 특성 분류
    \begin{itemize}
        \item MPKI 높음 $\Rightarrow$ 메모리 접근 많음 $\Rightarrow$ Memory-intensive VM
        \item MPKI 낮음 $\Rightarrow$ 메모리 접근 적음 $\Rightarrow$ Compute-intensive VM
    \end{itemize}
    \item 컴퓨팅 능력과 VM의 특성을 고려하여 VM과 PM(물리 서버)를 mapping함
    \begin{enumerate}
        \item VM은 compute-intensiveness의 내림차순, PM은 capability의 내림차순으로 정렬
        \item Most compute-intensive VM을 highest-capability PM에 할당 $\Rightarrow$ PM 순서 업데이트
        \item (2)를 반복하여 Compute-intensive VM을 모두 배치
        \item Memory-intensive VM은 최대한 분산하여 PM에 배치
    \end{enumerate}
    \item VM manager, PM manager, Info.manager를 이용하여 검증
\end{itemize}

\subsubsection*{Experimental Results}
\begin{itemize}
    \item 네 개의 실험군
    \begin{itemize}
        \item CS: Consolidation
        \item LB: Load-balancing
        \item LV: Load-balancing with VM characteristics
        \item AB: Ambient-aware VM allocation
    \end{itemize}
    \item 성능 (속도): AB $>$ LV $\approx$ LB $\gg$ CS
    \item 평균 온도: AB $>$ LB $\approx$ LV $>$ CS
    \item 최고 온도: AB $>$ CS $\leq$ LB $\approx$ LV
    \item 에너지 소모: LV $<$ LB $<$ AB $\ll$ CS
\end{itemize}

\subsection{Inline Data Deduplication Accelerator}

\subsubsection*{Data deduplication}
\begin{itemize}
    \item 데이터 중복을 없애 저장 장치에 효율적으로 저장하는 기술
    \item 필요성: 오늘날 연간 데이터 생산량은 매우 높으며(44ZB at 2020) 70\% 정도의 데이터가 중복 데이터
    \item Overall scheme
    \begin{enumerate}
        \item Chunking: 저장될 데이터 파일을 chunking ($>$ 16KB)
        \item Fingerprinting: 해시 함수 등을 이용하여 해시값 도출
        \item Indexing: Fingerprint를 이용하여 데이터를 찾아봄
        \item Writing: Redundant 데이터는 metadata만 저장, Unique 데이터는 chunk를 저장
    \end{enumerate}
    \item Hardware level
    \begin{itemize}
        \item GPU가 메인 메모리에 요청 $\rightarrow$ 데이터가 L2, L1 캐시, register file등을 거쳐 load $\rightarrow$ deduplication 수행
        \item Memory access가 굉장히 많음 (Fingerprinting $\approx$ $2^{25}$, Indexing $\approx$ $2^{52}$)
    \end{itemize}
    \item 장점과 단점
    \begin{itemize}
        \item 장점: 저장 장치에서 차지하는 용량 감소, 쓰기 연산 횟수가 감소하여 수명 연장
        \item 단점: 저장하는 과정이 오래 걸림
    \end{itemize}
    \item Processing-in-Memory (PIM)
    \begin{itemize}
        \item 비교적 간단한 연산(Fingerprinting, Indexing)을 메모리 내부에서 수행
        \item 시간 단축, 발열 억제가 관건
    \end{itemize}
\end{itemize}

\subsubsection*{Locating Deduplication Unit (DU)}
\begin{itemize}
    \item Base die
    \begin{itemize}
        \item 하단에 위치; 성능에 초점을 맞춘 chip
        \item 가장 뜨거운 PHY 부분에서 멀리 떨어진 곳에 DU 배치
    \end{itemize}
    \item Core die
    \begin{itemize}
        \item DRAM cell이 위치; Storage density에 초점을 맞춘 chip
        \item 각 chip마다 배치
        \item 접근 시간 단축 (장점), storage density는 감소 (단점) $\Rightarrow$ 실제 메모리 접근 시간은 5배 정도 증가
    \end{itemize}
\end{itemize}

\subsubsection*{Experimental Results}
\begin{itemize}
    \item Latency
    \begin{itemize}
        \item Base-die DU는 GPU보다 4.3 - 23.7\% 감소
        \item Core-die DU는 Base-die DU보다 4.2\% 감소; 단, 전체 실행 시간은 500\% 증가
    \end{itemize}
    \item 전력 소모: Base-die DU는 GPU보다 45.5\% 감소
    \item 온도: Threshold 온도를 넘지 않음
\end{itemize}
\newpage

\subsection{Multidimensional Databases Accelerator}

\subsubsection*{Motivation}
\begin{itemize}
    \item Conventional storage method: 복잡한 데이터 할당을 방지하기 위해 row-oriented 방식
    \item 데이터를 load할 때는 필요하지 않은 데이터도 읽힐 수 있음 $\Rightarrow$ 비효율적
    \item 데이터를 load할 때 CPU가 column-oriented 방식으로 reorganize하는 방법
    \begin{itemize}
        \item SSD $\Rightarrow$ 메모리로 raw data, 메모리 $\Rightarrow$ 프로세서로 reorganize 필요한 데이터 $\Rightarrow$ 프로세서에서 메모리로 reorganized 데이터
        \item CPU 기반 시스템에서는 시간이 오래 걸리고 에너지 소모가 큼
    \end{itemize}
    \item SSD에서 데이터가 전송되기 직전 IDRA(In-storage Data Reorg. Accelerator)가 reorganize 수행하는 아이디어
\end{itemize}

\subsubsection*{Dimension Slice}
\begin{itemize}
    \item IDRA에서 사용되는 data reorganization 기법
    \item 하나의 차원에 존재하는 데이터를 쪼개어서 저장하는 방법
    \item Early-stopping이 가능 $\Rightarrow$ 빠른 스캔 가능
    \item Unused zero padding bit으로 인해 fault가 발생 가능 $\Rightarrow$ Lower bit으로 채우도록 수정
\end{itemize}

\subsubsection*{IDRA Scheme}
\begin{enumerate}
    \item 1x2 DEMUX: Data reorganization 여부에 따라 IDRA 실행 여부 판별
    \item Configuration Register: App에 따른 splitting method를 구성
    \item Reorganization Module: Data reorganization을 수행
    \item Output Buffer: 2개의 output buffer를 사용; 한 buffer가 데이터를 모으는 동안 다른 buffer는 export
    \item Output Module: 원하는 크기만큼 output data가 모이면 export
\end{enumerate}
\begin{figures}
    \fig{idra-structure.PNG}{.6}
\end{figures}

\subsubsection*{Experimental Results}
\begin{itemize}
    \item 성능 (데이터 재구성 시간)
    \begin{itemize}
        \item IDRA 기반 시스템은 Baseline 기반 시스템보다 78.6\% 단축
        \item Baseline 기반 시스템의 데이터 로드 시간 $\approx$ IDRA 기반 시스템의 데이터 로드 + 재구성 시간
    \end{itemize}
    \item 에너지 소모: IDRA 기반 시스템이 Baseline 기반 시스템보다 30.3\% 단축
\end{itemize}

\subsection{3D Stacked CPU Characterization}

\subsubsection*{TSV-based 3D integration}
\begin{itemize}
    \item 공정 미세화 진행 $\Rightarrow$ wire delay overhead 증가 $\Rightarrow$ (TSV-based) 3D integration으로 해결 가능
    \item 장점
    \begin{itemize}
        \item 큰 die가 여러 작은 die로 나누어질 수 있음 $\Rightarrow$ 수율과 비용 절감
        \item Heterogeneous integration을 가능케 함 $\Rightarrow$ Flexibility 증가
    \end{itemize}
    \item 단점: 전력 밀도 증가, 열 발산 저하 $\Rightarrow$ 2D chip에 비해 온도가 높음
    \item TSV 기반 3D CPU가 항상 2D CPU보다 성능이 높다고 단정할 수 없음
\end{itemize}

\subsubsection*{Dynamic Thermal Management by Intel}
\begin{itemize}
    \item Power Limit 1(PL1): TDP(Thermal Design Power), Base clock $f$
    \item Power Limit 2(PL2): 최대 전력, Design clock $f$
    \item Tau: PL2가 유지되는 시간을 나타내는 변수
    \item 프로세서의 $f$는 Tau 시간동안 PL2를 유지하고, 이후 PL1으로 떨어져 유지
    \item 문제점
    \begin{itemize}
        \item Fluctuation at PL2 phase: 전력 부족, 다른 core의 영향 등
        \item Safe temperature에서도 PL1 유지: 배터리 수명, skin 온도 등
    \end{itemize}
    \item Objective: Intel의 DTM과 DVFS 등이 3D CPU의 온도 관리에 효과적인가?
\end{itemize}
\begin{figures}
    \fig{intel-dtm-graph.PNG}{.55}
\end{figures}

\subsubsection*{Experimental Results}
\begin{itemize}
    \item 실험군
    \begin{itemize}
        \item 2D-DT: DTM 제거 + heat pipe + 작은 fan (Intel\textsuperscript{\textregistered} Dynamic Tuning Technology 비활성화)
        \item 3D-PL1: PL1 + heat pipe
        \item 3D-DVFS: DVFS + PL1 비활성화 + heat pipe
        \item 3D-DT: DTM 제거 + PL1 비활성화 + heat pipe + 외부 cooling fan
    \end{itemize}
    \item 성능: 3D-DT $>$ 3D-DVFS $>$ 2D-DT $\gg$ 3D-PL1
    \item 전력 소모: 3D-PL1 $<$ 3D-DVFS $\approx$ 3D-DT $<$ 2D-DT
    \item 평균 $f$: 2D-DT $\gg$ 3D-DT $\geq$ 3D-DVFS $>$ 3D-PL1
    \item 에너지 소모: 3D-DVFS $>$ 3D-PL1 $\approx$ 3D-DT (Fan 전력 미포함)
    \item 최고 온도: 3D-DVFS $>$ $T_\mathrm{th}$ $\approx$ 3D-PL1 $>$ 3D-DT $\geq$ 2D-DT
    \item 평균 온도: $T_\mathrm{th}$ $\approx$ 3D-DVFS $\gg$ 3D-PL1 $\geq$ 3D-DT $\geq$ 2D-DT
\end{itemize}

\subsubsection*{Conclusion}
\begin{itemize}
    \item Intel 제시 방법: 제 성능을 내지 못하게 제한하여 온도 문제를 제대로 해결하지 못함
    \item DVFS: 성능 저하도 발생하면서, 온도가 threshold를 초과함
    \item 외부 냉각 방식이 동원되어도 3D CPU의 온도 문제는 완전히 해결되지 못함
\end{itemize}

\subsection{References}
\begin{enumerate}
    \item Young-Ho Gong, Jae Jeong Yoo, and Sung Woo Chung, ``Thermal Modeling and Validation of A Real-World Mobile AP'', IEEE Design \& Test, vol.35, no.1, pp.55-62, February 2018.
    \item Young Geun Kim, Minyong Kim, Jae Min Kim, and Sung Woo Chung, ``M-DTM: Migration-based Dynamic Thermal Management for Heterogeneous Mobile Multi-core Processors'', Design, Automation and Test in Europe Conference (DATE 2015), Grenoble, France, March 2015.
    \item Hyung Beom Jang, Jinhang Choi, Ikroh Yoon, Sung-Soo Lim, Seungwon Shin, Naehyuck Chang, and Sung Woo Chung, ``Exploiting Application/system-dependent Ambient Temperature for Accurate Microarchitectural Simulation'', IEEE Transactions on Computers, vol.62, no.4, pp.705-715, April 2013.
    \item Young Seo Lee, Kyung Min Kim, Ji Heon Lee, Jeong Hwan Choi, and Sung Woo Chung, ``A High-Performance Processing-in-Memory Accelerator for Inline Data Deduplication'', IEEE International Conference on Computer Design (ICCD 2019), Abu Dhabi, UAE, November 2019.
    \item Seon Young Kim, Yon Dohn Chung, and Sung Woo Chung, ``IDRA: An In-storage Data Reorganization Accelerator for Multidimensional Databases'', IEEE Embedded Systems Letters, accepted.
    \item Ji Hun Kwon, Seung Hun Choi, and Sung Woo Chung, ``Characterizing the On-chip Temperature of an Off-the-shelf TSV-based 3D Stacked CPU'', IEEE Intersociety Conference on Thermal and Thermalmechanical Phenomena in Electronic Systems (ITherm 2021), June 2021.
\end{enumerate}
