\section{Microarchitecture-Driven Power Management}

\subsection{Power Consumption}

\subsubsection*{Dynamic Power}
\begin{itemize}
    \item 동적 전력. Chip이 동작할 때 소모되는 전력
    \begin{equation}
        P_{\mathrm{dyn}} = \alpha \cdot C_L \cdot V_{\mathrm{dd}}^2 \cdot f
    \end{equation}
    \item $C_L$: Capacitance
    \begin{itemize}
        \item 수용 가능한 전하의 총량
        \item 사용되는 gate의 수(fan-out), 전선 길이, 트랜지스터 크기 등
        \item 너무 낮으면 0과 1을 구별할 수 없으므로 한계가 있음
    \end{itemize}
    \item $V_{\mathrm{dd}}$: Supply voltage
    \begin{itemize}
        \item 세대가 지나며 많이 낮아지고 있음
        \item 반드시 $V_{\mathrm{th}}$(threshold voltage)보단 높아야 함
    \end{itemize}
    \item $f$: Frequency. 낮으면 performance가 떨어지므로 낮추기 어려움. 최근에는 계속 증가
    \item Switching Power, Clock Power 등이 있음
\end{itemize}

\subsubsection*{Leakage Power}
\begin{itemize}
    \item 누설 전력, 정적 전력(static power). Chip이 동작하지 않아도 소모되는 전력
    \begin{equation}
        P_{\mathrm{leak}} = I_{\mathrm{leak}} \cdot V_{\mathrm{dd}}
    \end{equation}
    \item $\log P_{\mathrm{leak}} \varpropto T$ $\Rightarrow$ 공정 미세화가 진행되면 누설전력 급증
    \item 공정미세화로 인해 Subthreshold Leakage가 dominant $\Rightarrow$ 대부분 $P_{\mathrm{leak}}$을 줄이는 기술은 여기에 집중
    \begin{figures}
        \fig{leakage-power-components.PNG}{.35}
    \end{figures}
    \item 공정미세화가 진행되면서 누설전력이 급증하는 추세임
    \item Core는 입력이 있을때만 gate가 작동하지만, cache는 입력이 없어도 F/F 등에 데이터 저장 $\Rightarrow$ 누설 발생
\end{itemize}
\newpage

\subsection{Low-Power Interconnect}

\subsubsection*{Interconnect and Power Consumption}
\begin{itemize}
    \item Chip간의 interconnect, 즉 버스는 전력 소모의 주 원인
    \begin{itemize}
        \item PC는 대부분 버스 구조를 갖는다
        \item 모바일은 버스 구조를 사용하지 않고, point-to-point 구조 $\Rightarrow$ chip size가 큼
    \end{itemize}
    \item Chip의 성능 향상 $\Rightarrow$ 트랜지스터 증가, 전선의 밀도 증가 $\Rightarrow$ $C_L$ 증가
\end{itemize}

\subsubsection*{Bus Encoding}
\begin{itemize}
    \item 버스 인코딩 방식을 바꾸어 switching activity를 줄이는 방법
    \item 두 값 사이의 hamming distance가 wire 수의 절반 초과 $\Rightarrow$ 대부분 버스가 switch 됨
    \begin{itemize}
        \item 예: 0000 $\rightarrow$ 1110: Hamming Distance는 3
    \end{itemize}
    \item Bus Inversion
    \begin{itemize}
        \item Hamming Distance가 크면 그 다음 수를 invert시킴
        \item 0000 $\rightarrow$ 1110보다 0000 $\rightarrow$ 0001이 switching activity가 더 적다
        \item Invert했음을 표시하는 flag bit가 필요함. 1bit만 필요하므로 이득이 될 수 있음
        \item Inversion 과정에서 overhead 발생 $\Rightarrow$ 느리게 동작하는 storage 등에서 사용
        \item Hamming distance 계산 과정, En/decode 과정에서 latency가 발생한다는 단점
    \end{itemize}
    \item Gray Code Encoding
    \begin{itemize}
        \item 다음 수로 넘어갈 때 단 1bit만 바뀌는 인코딩
        \item 예: $10 = 1111_2$, $11 = 1110_2$
        \item 장점: Array encoding 등 순차적인 count의 관여가 많으면 동적 전력 감소
        \item 단점: Switch가 많을수록 에러 확률이 증가하여 noise에 취약함. 순차적인 count가 아니면 오히려 손해
    \end{itemize}
    \item One-Hot Encoding
    \begin{itemize}
        \item 데이터를 표현할 때 단 하나의 1만 사용됨
        \item 장점: 임의의 두 수 간에 switch 수는 많아야 2회 $\Rightarrow$ 동적 전력 소모량 감소, Error detection 쉬움
        \item 단점: Bit수가 지나치게 늘어남 $\Rightarrow$ En/decoder이 복잡, Too much cost
    \end{itemize}
\end{itemize}

\subsubsection*{Crosstalk}
\begin{itemize}
    \item Chip의 미세화 $\Rightarrow$ Wire간의 거리가 감소 $\Rightarrow$ 간섭(crosstalk) 발생 가능성 증가
    \item 간섭이 발생하면 switch에서의 delay, error 등의 발생 가능성 증가
    \item 해결 방법: 전류가 흐르는 wire 사이에 전류가 흐르지 않는 shield wire 배치
\end{itemize}

\subsubsection*{Low Swing Buses}
\begin{figures}
    \fig{low-swing-buses.PNG}{.6}
\end{figures}
\begin{itemize}
    \item 버스에서 정보를 전달할 때 더 낮은 전압으로 전달하는 방법
    \item Differential Signaling: 전압이 너무 낮아 detection 힘듦 $\Rightarrow$
        신호를 원래 신호와 역전시킨 신호로 분리한 후 비교하여 차이를 늘림 $\Rightarrow$ 비교적 detection 쉬움
    \item 장점: $P_\mathrm{dyn} \varpropto V_{\mathrm{dd}}^2$ $\Rightarrow$ 동적 전력이 감소, Crosstalk 예방, 전압폭 감소 $\Rightarrow$ latency 감소
    \item 단점: 부가적인 En/decoder 필요, Noise에 취약
    \item SRAM, DRAM 등에서 사용, North Bridge, South Bridge 등에서 미사용
\end{itemize}

\subsubsection*{Bus Segmentation}
\begin{figures}
    \fig{bus-segmentation.PNG}{.6}
\end{figures}
\begin{itemize}
    \item A $\rightarrow$ C로 보낼 때 A $\rightarrow$ F로 보낼 수 있을 정도의 전압을 걸어야 함 $\Rightarrow$ 전력 낭비
    \item 버스를 여러 segment로 분할, segment 간에는 traffic을 조절하는 link(F/F)로 연결: Register insertion
    \item Link에 위치한 F/F에서 신호를 계속 증폭시켜줌
    \item 장점: Short path에서는 전력 소모, latency 감소 효과
    \item 단점: Long path에서는 전력 소모, latency가 오히려 증가
    \item 주로 Mobile AP CPU에서 사용됨
\end{itemize}

\subsubsection*{Network On Chip (NOC)}
\begin{figures}
    \fig{network-on-chip.PNG}{.4}
\end{figures}
\begin{itemize}
    \item 각 Processing unit을 simple CPU router에 붙여서 운용
    \item Intel의 high-end server CPU에 사용됨
    \item P2P 구조보다는 성능이 낮지만 hardware cost는 낮음
\end{itemize}
\newpage

\subsection{Low-Power Memories}

\subsubsection*{Memories and Power Consumption}
\begin{itemize}
    \item Memory에 적용되는 기술들은 RAM, ROM뿐만 아니라 캐시에도 적용될 수 있음
    \item Types of Memories (빠르고 가격이 비싼 순위)
    \begin{enumerate}
        \item Register: F/F의 집합
        \item Cache: SRAM. 지속적인 전원 공급 필요 $\Rightarrow$ 누설 전력 큼
        \item Main memory: DRAM
        \item Storage: SSD, HDD, Magentic Tape, \ldots
    \end{enumerate}
    \item 전력 소모를 줄이기 위한 두 가지 접근
    \begin{itemize}
        \item Memory 접근당 전력 소모를 줄임
        \item Memory 접근 횟수를 줄임
    \end{itemize}
\end{itemize}

\subsubsection*{Splitting Memories into Smaller Sub-Systems(Bank)}
\begin{itemize}
    \item 메모리를 여러 개의 bank로 나누고 필요한 메모리가 있는 bank만 활성화
    \item Main Memory는 여러 bank로 나누어져 있음
    \item Compiler과 OS는 데이터를 최소한의 bank로 나눔 $\Rightarrow$ 다른 bank가 power down하여 전력 소모 줄임
    \item 단점: Data가 어느 bank에 있는지 알아내면서 latency 발생, Decoder $\cdot$ mux의 필요로 cost 증가
    \item Banked Cache (Image Below)
    \begin{itemize}
        \item Cache address: Tag - Index - Offset의 순서
        \item Index로 row 탐색 $\Rightarrow$ Tag로 hit/miss 판단 $\Rightarrow$ Offset으로 data 전달
        \item Bank selection bits 삽입, tag의 bits를 줄임 $\Rightarrow$ Bank selection
    \end{itemize}
    \begin{figures}
        \fig{banked-cache.PNG}{.4}
    \end{figures}
    \item Partitoned Power-Aware Instruction Cache
    \begin{itemize}
        \item 사용할 때는 bank를 enable하고 사용하지 않을때는 low power mode(DVFS)로 전환
        \item Prediction을 기반으로 이루어짐
    \end{itemize}
\end{itemize}
\newpage

\subsubsection*{Augmenting the Memory Hierachy with Specialized Cache Structure}
\begin{itemize}
    \item L0 Cache (Filter Cache)
    \begin{itemize}
        \item CPU와 L1 Cache 사이에 작은 L0 Cache를 삽입. L0 Cache에 먼저 접근, miss 시 L1에 접근
        \item L0 Cache는 app의 working set의 대부분을 저장 $\Rightarrow$ 대부분의 memory access를 걸러냄
        \item 전력 소모는 줄어드나 cost는 증가. 단, L0 miss \& L1 hit 발생 시 전력 소모 증가
    \end{itemize}
    \item Filter TLB (Translation Lookahead Buffer)
    \begin{itemize}
        \item TLB: Page table의 cache와 같은 개념
        \item Page table: Virtual address와 Physical address의 mapping. Main memory 내에 위치
    \end{itemize}
    \item Scratch Pad Memory (SPM)
    \begin{itemize}
        \item SPM: Main memory처럼 동작하나 작은 용량의 SRAM($<$1MB)으로 이루어짐
        \item Main memory는 cache와 달리 hit/miss가 없음. 주소로 접근하면 100\% 원하는 data를 찾음
        \item 장점: 전력 소모 감소, latency 감소
        \item 단점: High cost
    \end{itemize}
    \item Trace Cache
    \begin{itemize}
        \item L1 cache에서 instruction을 compile된 순서로 저장하지 않고 실행된 순서로 저장 (address는 불변)
        \item Loop에서의 반복 실행이 발생하면 instruction cache hit rate이 높아짐
        \item Instruction cache access를 줄여 전력 소모가 감소함
        \item Instruction cache hit rate: $\approx$ 98 - 99\% $\Rightarrow$ $\approx$ 100\%
    \end{itemize}
    \item Dynamic Direction Prediction-Based Trace Cache
    \begin{itemize}
        \item Instruction을 가져올 때 Branch Prediction을 이용한 방식
        \item Branch prediction 과정의 latency로 인해 잘 사용되지 않음
    \end{itemize}
    \item Selective Trace Cache
    \begin{itemize}
        \item Hot Trace: 자주 실행된 trace $\Rightarrow$ Trace cache에 넣어둠
        \item 컴파일러가 branch instruction 뒤에 hot trace에 대한 힌트 삽입
        \item ISA 표준 개정 필요 $\Rightarrow$ Too much overhead $\Rightarrow$ 안 쓰임
    \end{itemize}
\end{itemize}

\subsection{Low-Power Caches}

\subsubsection*{Caches and Power Consumption}
\begin{figures}
    \fig{cache-structure.PNG}{.6}
\end{figures}
\begin{itemize}
    \item Cache의 작동 방식
    \begin{enumerate}
        \item Decoder가 cache address의 index 값을 받음 $\Rightarrow$ Word lines 활성화 $\Rightarrow$ Bit lines 뽑음
        \item Sense Amps: Low power swing을 사용하기 때문에 신호를 다시 증폭시켜야 함
        \item Comparators: Tag Comparison이 이루어짐 $\Rightarrow$ Hit/miss 판별
        \item Hit인 경우 데이터 내보냄, miss인 경우 내보내지 않음
    \end{enumerate}
    \item 전체 전력 소모의 절반 이상이 cache일 정도로 전력 소모가 큼
\end{itemize}

\subsubsection*{Dynamically Resizable Instruction (DRI)}
\begin{itemize}
    \item Cache의 사이즈를 필요한 만큼만 늘리고 줄여서 사용하는 방식
    \item $V_{\mathrm{dd}}$를 gating하여 cache의 일부분을 활성화/비활성화함
    \item App의 cache-miss 패턴을 분석하는 H/W profiler를 이용하여 활성화 여부 결정, Miss rate가 일정 threshold 초과하면 활성화
    \item 장점: 동적 전력 소모 감소. $I_{\mathrm{leak}}$이 감소하여 누설 전력 소모 감소 효과도 있음
    \item 단점
    \begin{itemize}
        \item 비활성화 시 데이터 휘발
        \item App history 분석, Cache profiling: Overhead 발생
        \item Reactivate 과정에서 전력 소모/latency 등 발생
    \end{itemize}
\end{itemize}

\subsubsection*{Cache Decay}
\begin{itemize}
    \item 각 cache line을 활성화/비활성화하여 사용하는 방식
    \item 일정 시간동안 no access $\Rightarrow$ Cache line을 sleep mode로 전환(turn off), 재access시 cache line 재활성화
    \item 장점: 동적 전력과 누설 전력이 모두 감소
    \item 단점
    \begin{itemize}
        \item 비활성화 시 데이터 휘발
        \item App history 분석, Cache profiling: Overhead 발생
        \item Decay 과정에서 latency 발생
        \item Power line마다 gating하는 것은 매우 큰 overhead 유발
    \end{itemize}
\end{itemize}

\subsubsection*{Drowsy Cache}
\begin{itemize}
    \item 사용하지 않는 cache를 turn-off(power gating)하지 않고, DVFS를 통해 low power mode로 전환
    \item Wake penalty: Low power mode $\rightarrow$ Nominal mode로 전환할때 시간이 걸림. 70nm 공정에선 1 cycle
    \item 단점: DVFS로 $V_{\mathrm{dd}}$가 $V_{\mathrm{th}}$ 근처까지 줄어들면 error가 발생할 수 있음
\end{itemize}

\subsubsection*{Dead Block Elimination}
\begin{itemize}
    \item App 실행 중 이미 사용이 완료된 block(dead block)을 power-down 해버리는 방식
    \item Compiler가 dead block hint 등 주어야 함 $\Rightarrow$ ISA 표준 개정 필요 $\Rightarrow$ Too much overhead
\end{itemize}
\newpage

\subsection{Adaptive Instruction Queue}
\begin{itemize}
    \item Instruction queue가 빠르게 실행되지 않으면 queue size를 줄임
    \item 여러가지 heuristic algorithm으로 판별함
\end{itemize}

\subsection{DVFS (Dynamic Voltage Frequency Scaling)}

\subsubsection*{Necessity of Microcontroller}
\begin{itemize}
    \item Voltage는 frequency보다 훨씬 느리게 변함
    \item $f$를 먼저 증가시키면 $V_{\mathrm{dd}}$가 $f$의 증가 속도를 못 따라감
        $\Rightarrow$ $V_{\mathrm{dd}}-f$ pair 깨짐 $\Rightarrow$ Chip 오동작 가능
    \item $V_{\mathrm{dd}}$가 증가한 것을 확인한 후 $f$를 올려야 함 $\Rightarrow$ Microcontroller의 task
\end{itemize}

\subsubsection*{Hardware Scheme and Software Scheme}
\begin{itemize}
    \item Hardware Scheme: $\mu$s 단위로 실행, Page보다 작은 단위로 관리
    \item Software Scheme: ms 단위로 실행, Page 단위로 관리
\end{itemize}

\subsubsection*{Definition}
\begin{itemize}
    \item Clock frequency를 보장할 수 있을 정도로 $V_\mathrm{dd}$를 낮춤 $\Rightarrow$ 전력 소모 감소
    \item Performance target을 달성할 수 있을 정도로 $f$를 낮춤 $\Rightarrow$ 전력 소모 감소
\end{itemize}

\subsubsection*{S/W Scheme Based on H/W primitives}
\begin{itemize}
    \item H/W: Power mgmt chip 등의 microcontroller로 $V_{\mathrm{dd}}$, $f$ 적용하는 primitive(firmware) 제공
    \item S/W: Management 담당
    \begin{itemize}
        \item Frequency를 제공하는 voltage regulator과 같은 chip은 CPU의 바깥쪽에 존재 $\Rightarrow$ I/O가 수반됨
        \item Voltage regulator과 같은 chip을 CPU 안으로 가져오면 cost가 매우 커짐 $\Rightarrow$ 손해
        \item CPU 밖에서 frequency control이 이루어지면 I/O가 수반 $\Rightarrow$ H/W에서 수행하더라도 어차피 느림
        \item 나중에 기술이 발전하여 CPU 안으로 가져와도 cost가 크게 늘지 않으면 H/W scheme이 될 수 있음
    \end{itemize}
\end{itemize}

\subsubsection*{System Load (CPU utilization)}
\begin{itemize}
    \item Heavy Load: High $f$, $V_{\mathrm{dd}}$, $P$, Low memory access
    \begin{itemize}
        \item Hard real-time 작업: Strictly 실시간으로 수행되어야 함. 예: 원자력 발전소 등
    \end{itemize}
    \item Soft Load: Low $f$, $V_{\mathrm{dd}}$, $P$, High memory access
    \begin{itemize}
        \item Soft real-time 작업: 실시간으로 수행되되, 일정 시간 내에만 수행되면 됨. 예: Multimedia apps 등
    \end{itemize}
\end{itemize}

\subsubsection*{Pros and Cons}
\begin{itemize}
    \item Pros
    \begin{itemize}
        \item $f$가 증가하면 반드시 $V_\mathrm{dd}$가 증가하여야 함. $f$가 감소하면 $V_\mathrm{dd}$ 낮출 수 있음
        \item $P \varpropto V_\mathrm{dd}^2 f \varpropto f^3$이므로 전력 소모가 $f^3$에 비례하여 감소? $\Leftarrow$ No
        \item $P_{\mathrm{dyn}} \varpropto f^3$, 그러나 DVFS의 목적은 $P_{\mathrm{leak}}$의 감소
    \end{itemize}
    \item Cons: 앞으로 수행될 작업을 예측하기 힘듦 (unpredictable)
\end{itemize}
\newpage

\subsection{Microprocessor and Peripheral Power Management}

\subsubsection*{Misconception of DVFS}
\begin{itemize}
    \item Deadline안에 전력을 가장 효율적으로 소모하여 작업을 수행하는 방법이 아님!
    \item 작업 수행 시간이 증가 $\Rightarrow$ 주변(peripheral) 기기는 계속 작동 $\Rightarrow$ 전력 소모 오히려 증가 가능
    \item 동적 전력은 감소할 수 있으나 누설 전력은 오히려 증가함
\end{itemize}

\subsubsection*{Making Use of Memory Wall}
\begin{itemize}
    \item Memory access는 시간이 긺 $\Rightarrow$ Memory 값이 필요한 Task 3은 늦게 실행됨
    \item Task 2가 아무리 일찍 끝나도 memory access가 끝나기 전까지는 Task 3 실행 불가 $\Rightarrow$ Stall
    \item Memory access가 많은 application은 $f$를 낮추어도 performance의 차이가 거의 없음
    \item CPU utilization 확인 (낮으면 memory access 많음) $\Rightarrow$ DVFS 적극 사용
\end{itemize}
\begin{figures}
    \fig{memory-wall.PNG}{.55}
\end{figures}

\subsubsection*{DVFS in Multiple Clock Domain Architecture}
\begin{itemize}
    \item GALS: Globally Asynchronous, Locally Synchronous
    \begin{itemize}
        \item Synchronous Logic: Logic 내의 모든 F/F register가 하나의 clock으로 동작
        \item Asynchronous Logic: Logic 내의 모든 F/F register가 하나의 clock으로 동작하는 것은 아님
    \end{itemize}
    \item 여러 CPU core를 여러 domain으로 나누고, 각 domain이 각기 다른 clock으로 동작함
    \item 장점: CPU domain간 전력 소모가 다름 $\Rightarrow$ 신호를 더 작은 영역에 생성
    \item CPU에서는 GALS를 많이 사용
    \item GPU는 대부분 하나의 task에 대해 전체 core가 같이 돎 $\Rightarrow$ Load가 거의 비슷 $\Rightarrow$ async로 돌 이유가 없음
\end{itemize}

\subsubsection*{Resource Hibernation}
\begin{itemize}
    \item 유휴 상태의 component를 power down하여 전력 소모 감소
    \item 디스크, 네트워크 interface, 디스플레이 등에서 많이 사용
\end{itemize}
\newpage

\subsection{Commercial Systems}

\subsubsection*{Energy-Flexibility(Generality) Gap}
\begin{itemize}
    \item Power과 performance 간의 tradeoff
    \item Dedicated H/W: High energy efficiency, Low flexibility
    \item Embedded Processors: Low energy efficiency, High flexibility
\end{itemize}
\begin{figures}
    \fig{energy-flexibity-gap.PNG}{.5}
\end{figures}

\subsection{Emerging Technologies}

\subsubsection*{Soft Error}
\begin{itemize}
    \item 같은 유형의 에러가 다시 발생하지 않을수도 있는 에러 ($\approx 10^{-7}$)
    \item $V_{\mathrm{dd}}$ 감소 $\Rightarrow$ Soft error 발생 확률 증가
    \item $\alpha$ 입자, 중성자 등으로 인해 발생
    \item 해결 방안
    \begin{itemize}
        \item Error checking code: Pairity bit, SEC-DED (Single Error Correction-Double Error Detection)
        \item Thread duplication $\Rightarrow$ Too much overhead $\Rightarrow$ 군수, 우주, 원자력 등에서만 이용
        \item Idle Pipeline Utilization $\Rightarrow$ Resource conflict 발생 가능
    \end{itemize}
\end{itemize}

\subsubsection*{Process Variation}
\begin{itemize}
    \item 공정 미세화 $\Rightarrow$ Design Margin Decrease $\Rightarrow$ 너무 작아서 미세한 변화(latency 등)도 체감이 큼
    \item 불랑품 발생 가능성 증가 $\Rightarrow$ 수율 감소
    \item 목표 $f$가 달성되지 않으면 $V_{\mathrm{dd}}$를 높여서 출하하기도 함
    \item Within die variation: intra-die variation
    \item Die-to-die variation: inter-die variation
\end{itemize}
\begin{figures}
    \fig{process-variation.PNG}{.5}
\end{figures}
