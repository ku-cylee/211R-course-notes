\section{Microarchitecture-Driven Power Management}

\subsection{Power Consumption}

\subsubsection*{Dynamic Power}
\begin{itemize}
    \item 동적 전력. Chip이 동작할 때 소모되는 전력
    \item $P_{\mathrm{dyn}} = \alpha \cdot C_L \cdot V_{\mathrm{dd}}^2 \cdot f$
    \item $C_L$: Capacitance
    \begin{itemize}
        \item 수용 가능한 전하의 총량
        \item 사용되는 gate의 수(fan-out), 전선 길이, 트랜지스터 크기 등
        \item 너무 낮으면 0과 1을 구별할 수 없으므로 한계가 있음
    \end{itemize}
    \item $V_{\mathrm{dd}}^2$: Supply voltage
    \begin{itemize}
        \item 세대가 지나며 많이 낮아지고 있음
        \item 반드시 $V_{\mathrm{th}}$(threshold voltage)보단 높아야 함
    \end{itemize}    
    \item $f$: Frequency. 낮으면 performance가 떨어지므로 낮추기 어려움. 최근에는 계속 증가
    \item Switching Power, Clock Power 등이 있음
\end{itemize}

\subsubsection*{Leakage Power}
\begin{itemize}
    \item 누설 전력, 정적 전력(static power). Chip이 동작하지 않아도 소모되는 전력
    \item $P_{\mathrm{leak}} = I_{\mathrm{leak}} \cdot V_{\mathrm{dd}}$
    \item $\log P_{\mathrm{leak}} \varpropto T$ $\Rightarrow$ 공정 미세화가 진행되면 누설전력 급증
    \item 공정미세화로 인해 Subthreshold Leakage가 dominant $\Rightarrow$ 대부분 $P_{\mathrm{leak}}$을 줄이는 기술은 여기에 집중
    \figcmd{lower-power-computing/images/leakage-power-components.PNG}{.5}
    \item 공정미세화가 진행되면서 누설전력이 급증하는 추세임
    \item Core는 입력이 있을때만 gate가 작동하지만, cache는 입력이 없어도 F/F 등에 데이터 저장 $\Rightarrow$ 누설 발생
\end{itemize}
\pagebreak

\subsection{Low-Power Interconnect}

\subsubsection*{Interconnect and Power Consumption}
\begin{itemize}
    \item Chip간의 interconnect, 즉 버스는 전력 소모의 주 원인
    \begin{itemize}
        \item PC는 대부분 버스 구조를 갖는다
        \item 모바일은 버스 구조를 사용하지 않고, point-to-point 구조 $\Rightarrow$ chip size가 큼
    \end{itemize}
    \item Chip의 성능 향상 $\Rightarrow$ 트랜지스터 증가, 전선의 밀도 증가 $\Rightarrow$ $C_L$ 증가
\end{itemize}

\subsubsection*{Bus Encoding}
\begin{itemize}
    \item 버스 인코딩 방식을 바꾸어 switching activity를 줄이는 방법
    \item 두 값 사이의 hamming distance가 wire 수의 절반 초과 $\Rightarrow$ 대부분 버스가 switch 됨
    \begin{itemize}
        \item 예: 0000 $\rightarrow$ 1110: Hamming Distance는 3
    \end{itemize}
    \item Bus Inversion
    \begin{itemize}
        \item Hamming Distance가 크면 그 다음 수를 invert시킴
        \item 0000 $\rightarrow$ 1110보다 0000 $\rightarrow$ 0001이 switching activity가 더 적다
        \item Invert했음을 표시하는 flag bit가 필요함. 1bit만 필요하므로 이득이 될 수 있음
        \item Inversion 과정에서 overhead 발생 $\Rightarrow$ 느리게 동작하는 storage 등에서 사용
        \item Hamming distance 계산 과정, En/decode 과정에서 latency가 발생한다는 단점
    \end{itemize}
    \item Gray Code Encoding
    \begin{itemize}
        \item 다음 수로 넘어갈 때 단 1bit만 바뀌는 인코딩
        \item 예: $10 = 1111_2$, $11 = 1110_2$
        \item 장점: Array encoding 등 순차적인 count의 관여가 많으면 동적 전력 감소
        \item 단점: Switch가 많을수록 에러 확률이 증가하여 noise에 취약함. 순차적인 count가 아니면 오히려 손해
    \end{itemize}
    \item One-Hot Encoding
    \begin{itemize}
        \item 데이터를 표현할 때 단 하나의 1만 사용됨
        \item 장점: 임의의 두 수 간에 switch 수는 많아야 2회 $\Rightarrow$ 동적 전력 소모량 감소, Error detection 쉬움
        \item 단점: Bit수가 지나치게 늘어남 $\Rightarrow$ En/decoder이 복잡, Too much cost
    \end{itemize}
\end{itemize}

\subsubsection*{Crosstalk}
\begin{itemize}
    \item Chip의 미세화 $\Rightarrow$ Wire간의 거리가 감소 $\Rightarrow$ 간섭(crosstalk) 발생 가능성 증가
    \item 간섭이 발생하면 switch에서의 delay, error 등의 발생 가능성 증가
    \item 해결 방법: 전류가 흐르는 wire 사이에 전류가 흐르지 않는 shield wire 배치
\end{itemize}

\subsubsection*{Low Swing Buses}
\figcmd{lower-power-computing/images/low-swing-buses.PNG}{.5}
\begin{itemize}
    \item 버스에서 정보를 전달할 때 더 낮은 전압으로 전달하는 방법
    \item Differential Signaling: 전압이 너무 낮아 detection 힘듦 $\Rightarrow$
        신호를 원래 신호와 역전시킨 신호로 분리한 후 비교하여 차이를 늘림 $\Rightarrow$ 비교적 detection 쉬움
    \item 장점: $P_\mathrm{dyn} \varpropto V_{\mathrm{dd}}^2$ $\Rightarrow$ 동적 전력이 감소, Crosstalk 예방, 전압폭 감소 $\Rightarrow$ latency 감소
    \item 단점: 부가적인 En/decoder 필요, Noise에 취약
    \item SRAM, DRAM 등에서 사용, North Bridge, South Bridge 등에서 미사용
\end{itemize}

\subsubsection*{Bus Segmentation}
\figcmd{lower-power-computing/images/bus-segmentation.PNG}{.5}
\begin{itemize}
    \item A $\rightarrow$ C로 보낼 때 A $\rightarrow$ F로 보낼 수 있을 정도의 전압을 걸어야 함 $\Rightarrow$ 전력 낭비
    \item 버스를 여러 segment로 분할, segment 간에는 traffic을 조절하는 link(F/F)로 연결: Register insertion
    \item Link에 위치한 F/F에서 신호를 계속 증폭시켜줌
    \item 장점: Short path에서는 전력 소모, latency 감소 효과
    \item 단점: Long path에서는 전력 소모, latency가 오히려 증가
    \item 주로 Mobile AP CPU에서 사용됨
\end{itemize}

\subsubsection*{Network On Chip (NOC)}
\figcmd{lower-power-computing/images/network-on-chip.PNG}{.5}
\begin{itemize}
    \item 각 Processing unit을 simple CPU router에 붙여서 운용
    \item Intel의 high-end server CPU에 사용됨
    \item P2P 구조보다는 성능이 낮지만 hardware cost는 낮음
\end{itemize}
\pagebreak

\subsection{Low-Power Memories}

\subsubsection*{Approaches for Low-Power Memories}
\begin{itemize}
    \item Memory에 적용되는 기술들은 RAM, ROM뿐만 아니라 캐시에도 적용될 수 있음
    \item Types of Memories (빠르고 가격이 비싼 순위)
    \begin{enumerate}
        \item Register: F/F의 집합
        \item Cache: SRAM. 지속적인 전원 공급 필요 $\Rightarrow$ 누설 전력 큼
        \item Main memory: DRAM
        \item Storage: SSD, HDD, Magentic Tape, \ldots
    \end{enumerate}
    \item 전력 소모를 줄이기 위한 두 가지 접근
    \begin{itemize}
        \item Memory 접근당 전력 소모를 줄임
        \item Memory 접근 횟수를 줄임
    \end{itemize}
\end{itemize}

\subsubsection*{Splitting Memories into Smaller Sub-Systems(Bank)}
\begin{itemize}
    \item 메모리를 여러 개의 bank로 나누고 필요한 메모리가 있는 bank만 활성화
    \item Main Memory는 여러 bank로 나누어져 있음
    \item Compiler과 OS는 데이터를 최소한의 bank로 나눔 $\Rightarrow$ 다른 bank가 power down하여 전력 소모 줄임
    \item 단점: Data가 어느 bank에 있는지 알아내면서 latency 발생, Decoder $\cdot$ mux의 필요로 cost 증가
    \item Banked Cache (Image Below)
    \begin{itemize}
        \item Cache address: Tag - Index - Offset의 순서
        \item Index로 row 탐색 $\Rightarrow$ Tag로 hit/miss 판단 $\Rightarrow$ Offset으로 data 전달
        \item Bank selection bits 삽입, tag의 bits를 줄임 $\Rightarrow$ Bank selection
    \end{itemize}
    \item Partitoned Power-Aware Instruction Cache
    \begin{itemize}
        \item 사용할 때는 bank를 enable하고 사용하지 않을때는 low power mode(DVFS)로 전환
        \item Prediction을 기반으로 이루어짐
    \end{itemize}
\end{itemize}

\figcmd{lower-power-computing/images/banked-cache.PNG}{.5}
\pagebreak

\subsubsection*{Augmenting the Memory Hierachy with Specialized Cache Structure}
\begin{itemize}
    \item L0 Cache (Filter Cache)
    \begin{itemize}
        \item CPU와 L1 Cache 사이에 작은 L0 Cache를 삽입. L0 Cache에 먼저 접근, miss 시 L1에 접근
        \item L0 Cache는 app의 working set의 대부분을 저장 $\Rightarrow$ 대부분의 memory access를 걸러냄
        \item 전력 소모는 줄어드나 cost는 증가. 단, L0 miss \& L1 hit 발생 시 전력 소모 증가
    \end{itemize}
    \item Filter TLB (Translation Lookahead Buffer)
    \begin{itemize}
        \item TLB: Page table의 cache와 같은 개념
        \item Page table: Virtual address와 Physical address의 mapping. Main memory 내에 위치
    \end{itemize}
    \item Scratch Pad Memory (SPM)
    \begin{itemize}
        \item SPM: Main memory처럼 동작하나 작은 용량의 SRAM(<1MB)으로 이루어짐
        \item Main memory는 cache와 달리 hit/miss가 없음. 주소로 접근하면 100\% 원하는 data를 찾음
        \item 장점: 전력 소모 감소, latency 감소
        \item 단점: High cost
    \end{itemize}
    \item Trace Cache
    \begin{itemize}
        \item L1 cache에서 instruction을 compile된 순서로 저장하지 않고 실행된 순서로 저장 (address는 불변)
        \item Loop에서의 반복 실행이 발생하면 instruction cache hit rate이 높아짐
        \item Instruction cache access를 줄여 전력 소모가 감소함
        \item Instruction cache hit rate: $\approx$ 98\textasciitilde99\% $\Rightarrow$ $\approx$ 100\%
    \end{itemize}
    \item Dynamic Direction Prediction-Based Trace Cache
    \begin{itemize}
        \item Instruction을 가져올 때 Branch Prediction을 이용한 방식
        \item Branch prediction 과정의 latency로 인해 잘 사용되지 않음
    \end{itemize}
    \item Selective Trace Cache
    \begin{itemize}
        \item Hot Trace: 자주 실행된 trace $\Rightarrow$ Trace cache에 넣어둠
        \item 컴파일러가 branch instruction 뒤에 hot trace에 대한 힌트 삽입
        \item ISA 표준을 개정해야 함 $\Rightarrow$ Too much overhead $\Rightarrow$ 안 쓰임
    \end{itemize}
\end{itemize}
