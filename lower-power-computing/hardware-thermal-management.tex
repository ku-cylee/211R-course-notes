\section{Hardware Thermal Management}

\subsection{Introduction}

\subsubsection*{Temperature Issue Arise}
\begin{itemize}
    \item 과거에는 성능 향상이 제일 중요한 과제였음
    \item 복잡한 미세공정, 높은 clock frequency, Chip area의 감소 $\Rightarrow$ Power Density 증가
    \item Power $\uparrow$ $\Rightarrow$ Power density(전력 밀도) $\uparrow$ $\Rightarrow$ $T$ $\uparrow$ $\Rightarrow$ $P_{\mathrm{leak}}$ $\uparrow$ $\Rightarrow$ Power $\uparrow$: Loop
    \item Multi-core architecture $\Rightarrow$ Thermal problem이 일시적으로 해소
\end{itemize}

\subsubsection*{Importance of Temperature Issue}
\begin{itemize}
    \item 공정미세화의 진행 $\Rightarrow$ 전력 밀도는 exponentially 상승 중 (원자로 수준에 근접)
    \item 신뢰도(reliability) 문제 초래: 수명, 영구적 손상 등
    \item Microprocessor Design Cost
    \begin{itemize}
        \item Core의 개수, 복잡도 등은 processor의 온도에 큰 영향을 미침
        \item 이러한 earliest design stages에서 온도에 관한 고려가 이루어져야 함
    \end{itemize}
    \item 온도는 매우 복잡한 과정으로 예측되며, 환경 요인에 영향을 많이 받음 $\Rightarrow$ 예측 어려움
    \begin{itemize}
        \item 주변 공기, 주변 chip, thermal grease의 종류 등에 영향 받음
    \end{itemize}
\end{itemize}

\subsubsection*{Temperature Management and Power Management}
\begin{itemize}
    \item 온도와 전력은 관계가 깊지만 전력의 절약이 항상 온도 감소를 가져오는 것은 아님
    \begin{itemize}
        \item 예: Memory Bank 도입하여 전력 소모 감소 $\Rightarrow$ 면적 감소로 전력 밀도 증가 $\Rightarrow$ 온도 증가
    \end{itemize}
    \item Power mgmt: 온도가 낮은(= underutilized) 영역에 집중, 변화 속도 빠름
    \item Temperature mgmt: 온도가 높은(= heavily utilized) 영역에 집중, 변화 속도 느림
    \item 두 management에 같은 techniques 사용 가능, But policies are different
\end{itemize}

\subsection{Temperature Monitoring}

\subsubsection*{Importance of Temperature Monitoring}
\begin{itemize}
    \item Chip의 온도는 매우 짧은 순간에 변함 $\Rightarrow$ 정확한 온도 monitoring X $\Rightarrow$ chip 손상 가능
    \item 효과적인 DTM(Dynamic Thermal Management)에 필수적인 요소임
    \item 대부분 CPU는 적어도 10개의 온도 센서를 가지고 있음 $\Rightarrow$ 센서의 데이터는 OS-level에서 읽을 수 있음
    \item Cool Down Techniques
    \begin{itemize}
        \item Thermal throttling: Operation stop, Clock ``literally'' stops. (HW scheme)
        \item DVFS: Power decreases (SW scheme)
    \end{itemize}
\end{itemize}
\pagebreak

\subsubsection*{Importance of On-chip Sensor Placement}
\begin{itemize}
    \item 열원과 센서간의 거리가 조금만 떨어져도 온도의 측정값은 불확실
    \item 열이 많이 발생할 것으로 예상되는 곳에 센서를 배치해야 함
    \item 너무 많은 센서를 배치하면 cost가 증가
    \begin{itemize}
        \item 센서의 면적(작음)보다는 analog-digital converter(큼)로 인해 주로 증가
        \item 최근에는 analog-digital converter를 share하여 사용하기도 함
    \end{itemize}
    \item 제한된 개수의 센서를 최적의 위치에 배치하는 것이 주요 과제
\end{itemize}

\subsubsection*{Techniques of On-chip Sensor Placement}
\begin{itemize}
    \item K-means clustering
    \begin{itemize}
        \item $k$ sensors (clusters) \& $n$ hotspots (data points) $\Rightarrow$ 각 $k$에 대해 optimal point를 찾아냄
    \end{itemize}
    \item Dynamic selection method
    \begin{itemize}
        \item Interpolation을 이용하여 열원 위치를 짐작함
        \item 통신량이 많음 $\Rightarrow$ serious interconnect power overhead
    \end{itemize}
\end{itemize}

\subsubsection*{Temperature Estimation}
\begin{itemize}
    \item Predictive (= Proactive): $T_{\mathrm{th}}$에 도달하기 전에 미리 thermal mgmt 수행
    \item Reactive: $T_{\mathrm{th}}$에 도달하면 thermal mgmt 수행
    \begin{itemize}
        \item 전력을 소모하고 온도가 올라갈 때까지 delay가 있음, Late response $\Rightarrow$ 비효율적!
    \end{itemize}
    \item Instruction의 operator에 따라 사용하는 CPU가 다름을 이용하여 prediction 수행
\end{itemize}

\subsection{(Micro)Architectural Techniques}

\subsubsection*{Dynamic Thermal Managment (DTM)}
\begin{itemize}
    \item 온도의 급격한 증가(thermal excursion)를 monitoring하여 조절
    \item DVFS, Throttling (clock gating), Cooling solution등을 이용함
    \item Critical Temperature and Threshold Temperature
    \begin{itemize}
        \item $T_{\mathrm{threshold}}$: 넘지 않으면 좋으나 넘어도 큰 문제는 없는 온도. S/W(DVFS)가 manage
        \item $T_{\mathrm{critical}}$: 넘으면 심각한 문제가 발생할 수 있는 온도. H/W(throttling)가 manage
        \item $T_{\mathrm{critical}} > T_{\mathrm{threshold}}$
    \end{itemize}
    \item 최근 연구 주제: Costly throttling, 성능 저하 최소화
    \item DVFS나 Throttling을 통한 DTM은 필연적으로 delay 발생 (H/W: \textless 1$\mu$s, S/W: \textasciitilde 10ms)
    \begin{figures}
        \fig{lower-power-computing/images/dtm-by-dvfs.PNG}{.5}
    \end{figures}
    \item 주로 frequency를 낮추어 DTM 수행. 100MHz 단위의 조절 가능 (전압은 낮아지는데 시간이 오래걸림)
\end{itemize}

\subsubsection*{Techniques for Microprocessor Cores/General Applications}
\begin{itemize}
    \item Feedback Control (Skadron et al. [2003])
    \begin{itemize}
        \item Application behavior을 토대로 전압과 frequency 조절 $\Rightarrow$ 성능 저하 최소화
        \item 대부분의 DTM techniques에 이용될 수 있음
    \end{itemize}
    \item DTM Considering Fan Speed (Shin et al. [2009])
    \begin{itemize}
        \item 전력 소모와 온도의 co-optimization이 필요
        \item 예: 온도를 낮추기 위해 cooling fan speed up
        \begin{itemize}
            \item 팬 사용으로 인해 추가 전력 발생
            \item 팬 사용으로 cool down되어 누설 전력 감소
        \end{itemize}
        \item 두 요소 같이 고려해야 함
    \end{itemize}
    \item Frequency Selection for Multi-core Microprocessor (Mukherjee and Memik [2006b])
    \begin{itemize}
        \item Core가 여러 개인 경우 코어 간에 열이 발산됨
        \item Core에 최적 frequency 사용, 이웃한 core의 사용을 피함
    \end{itemize}
    \item Adapt3D (Coskun et al. [2009]): Chip을 stack한 형태의 multi-core에 위의 technique 적용
\end{itemize}

\subsubsection*{Techniques for On-chip Caches}
\begin{itemize}
    \item Cache는 동적 전력이 낮아 온도가 낮음 / 누설 전력은 계속 증가함 $\Rightarrow$ Chip에서 높은 누설 전력 비중 차지
    \begin{itemize}
        \item \texttt{ld}, \texttt{sd}만 cache 사용 (\textless 20\%)
    \end{itemize}
    \item 유휴 상태 Power Reduction: Gated $V_{\mathrm{dd}}$, DVFS 등
    \item 하나의 cache line을 여러 row로 분산 $\Rightarrow$ 전력 밀도 $\downarrow$ (John et al. [2005])
    \item Selective cache ways + Gated $V_{\mathrm{dd}}$ (Ku et al. [2005])
    \item 원래 온도가 낮은 영역이기 때문에 비효율적임
\end{itemize}

\subsection{Design Techniques}

\subsubsection*{Clustered Architecture}
\begin{itemize}
    \item Chaparro et al. [2005]
    \begin{itemize}
        \item 전통적인 프로세서를 partition하여 만든 여러 cluster로 구성
        \item 중앙화, 포트가 많은 구조 $\Rightarrow$ 포트가 적은 구조가 여러 개 분산된 구조
        \item 분산된 cluster들은 작기 때문에 더 높은 clock frequency 사용 가능 $\Rightarrow$ 성능 향상
        \item 칩의 동작을 넓게 분산시켜 열이 발산되나, 통신 overhead가 존재
        \item Bank hopping: 여러 bank를 구성하여 동작하지 않는 bank는 sleep $\Rightarrow$ cool down
        \item Trace cache를 이용하여 new trace를 가장 차가운 bank에 assign하도록 mapping 함수를 수정
    \end{itemize}
\end{itemize}

\subsubsection*{Replicated Functional Units}
\begin{itemize}
    \item Activity Migration (Heo et al. [2003])
    \begin{itemize}
        \item Register file, ALU 등 통상적으로 뜨거운 functional unit(FU)을 복제 (identical)
        \item 전체 pipeline을 복제하지 않음
        \item 일정 시간마다 identical unit으로 switch $\Rightarrow$ 다른 unit은 유휴 상태
        \item 최고 온도가 약 12$\cel$ 감소
        \item 면적 overhead가 발생
    \end{itemize}
    \newpage
    \item Migrating Computation (Skadron et al. [2003])
    \begin{itemize}
        \item 다른 FU를 사용하는 방법. Register file만 복사
        \begin{enumerate}
            \item 1차 register file이 DTM 실행 온도 도달
            \item Instruction issue stall $\Rightarrow$ Write-back 될 때까지 기다림
            \item 해당 instruction을 2차 register file 사용하여 실행
            \item 1차 register file 온도가 낮아지면 1차 register file로 복귀시킴
        \end{enumerate}
        \item Area overhead(Activity migration보다는 낮음)와 온도 간의 tradeoff
    \end{itemize}
\end{itemize}

\subsubsection*{Resizing Functional Units}
\begin{itemize}
    \item Resource Area Dilation (Powell and Vijaykumar [2007])
    \begin{itemize}
        \item Simple-RAD
        \begin{itemize}
            \item Hot FU 크기를 늘려 전력 밀도 $\Downarrow$
            \item Wire 길이가 늘어나므로 latency가 증가한다는 단점
        \end{itemize}
        \item Pipelined-RAD
        \begin{itemize}
            \item Hot FU의 크기를 늘리면서 unit을 pipeline 시킴 $\Rightarrow$ $f$ 감소하지 않음
            \item 열 제한이 있는 작업 처리량이 41-56\% 증가하였음
            \item ALU, register file등은 아날로그 $\Rightarrow$ noise에 민감
                $\Rightarrow$ 온도 센서 배치 힘듦, 정확도 낮음 $\Rightarrow$ 잘 사용되지 않음
        \end{itemize}
    \end{itemize}
\end{itemize}

\subsubsection*{Floorplanning}
\begin{itemize}
    \item Floorplanning: Chip의 배치 등의 설계도
    \begin{itemize}
        \item Thermal-aware floorplanning은 그 어떤 DVFS보다도 높은 성능을 보임
        \item Chip의 배치는 wire의 routing과도 직결됨 $\Rightarrow$ Latency에 영향
        \item Chip의 배치는 온도와도 직결됨
        \begin{itemize}
            \item 뜨거운 두 chip이 붙어있음 $\Rightarrow$ Low latency, High power density
            \item 뜨거운 두 chip이 떨어져 있음 $\Rightarrow$ High latency, Low power density
        \end{itemize}
    \end{itemize}
    \item HotFloorplan (Sankaranarayanan et al. [2005])
    \begin{itemize}
        \item 최고 온도, Chip 면적, Wire delay 등을 고려한 2D floorplanning tool
        \item min. $(\mathrm{Obj}) = (A + \lambda W)T$
        \begin{itemize}
            \item A: Chip 면적, T: 최고 온도, W: $\sum$ weighted wire 길이
        \end{itemize}
    \end{itemize}
    \item 3D Floorplanning
    \begin{itemize}
        \item 3D stakcing: 여러 die가 여러 layer를 구성
        \item Vertical 열 전도가 발생하므로 2D에 비해 열 발산이 힘듦 $\Rightarrow$ 열 문제가 더 심각
    \end{itemize}
\end{itemize}

\subsubsection*{OS/Complier-level Techniques}
\begin{itemize}
    \item 장점: Hardware overhead 감소
    \item 단점: 상태/정보 등을 관리하기 위한 추가 데이터 구조 필요
    \item Design issues: 성능, storage, 컴파일 시간 overhead를 줄여야 함
\end{itemize}
\newpage

\subsubsection*{Thermal-aware Task Scheduling}
\begin{itemize}
    \item Thermal-aware priority queue design (Kumar et al. [2006])
    \begin{itemize}
        \item Hot task에는 낮은 우선순위, Cool task에는 높은 우선순위 부여
        \item Task의 hot/cool 여부는 performance counter 값을 참고하여 판단
        \item 모든 task가 hot task일 경우 대응 불가 $\Rightarrow$ Failsafe: Clock gating
    \end{itemize}
    \item Thermal-aware load balancing and task migration (Merkel et al. [2006; 2008])
    \begin{itemize}
        \item Passive LB: Run queue의 길이를 가능한 한 같게 유지 $\Rightarrow$ 열 발산
        \item Active hot task migration
        \begin{itemize}
            \item Core에서 실행 중인 작업을 미리 다른 core로 옮김 (migrate)
            \item 실행중인 작업이 thermal emergency를 초래하는 것을 방지
        \end{itemize}
    \end{itemize}
    \item Task sequencing (Yang et al. [2008])
    \begin{itemize}
        \item Thermal behavior의 temporal correlations 고려
        \item Hot task와 cold task를 실행할 때 hot$\rightarrow$cool 순서가 cool$\rightarrow$hot 순서보다 온도가 덜 오름
        \item 열적 문제를 야기하지 않는 선에서 hottest task를 먼저 실행
        \item 3.25-4.70\% 정도의 성능 향상
    \end{itemize}
    \item Task scheduling for 3D microprocessors (Zhou et al. [2008])
    \begin{itemize}
        \item Super core: 2D 위치가 같은, 수직으로 나열된 core의 집합, Super Task: 같이 schedule된 task의 집합
        \item 3D는 수직 방향으로도 열전도 $\Rightarrow$ 온도가 제일 높은 core와 전력 소모가 제일 높은 core는 다를 수 있음
        \item 기존 방식은 주로 온도가 제일 높은 core를 냉각, 3D에서는 전력 소모가 제일 높은 core를 냉각
    \end{itemize}
    \item Embedded/Real-time processors
    \begin{itemize}
        \item For embedded: 에너지 reduction이 목표
        \item Real-time: QoS(deadline 내에 task 완료)를 보장하는 것이 목표
        \item 온도, energy/QOS, 성능 등을 같이 고려해야 함
    \end{itemize}
\end{itemize}

\subsubsection*{Compiler-directed Techniques}
\begin{itemize}
    \item 장점: H/W support 불필요 $\Rightarrow$ H/W overhead가 적음
    \item 단점: Run-time에 프로그램의 dyanmic behavior을 capture할 수 없음 $\Rightarrow$ 잘 사용되지 않음
    \item Compile-time에 VLIW(Very Long Instruction Word) 프로세서가 사용됨
    \begin{itemize}
        \item VLIW에서 프로세서의 therm. behavior는 inst. bundle 내의 inst. 배열, inst. bundle의 scheduling과 관련
    \end{itemize}
    \item Load balancing and IPC tuning by compilers (Mutyam et al. [2006])
    \begin{itemize}
        \item Load balancing: 동적 전력을 FU 간에 가능한 한 균일하게 분산
        \item IPC tuning: Loop에서 (ALU 등) 정해진 개수만 활성화되어 있으면 사용하지 않는 FU를 끔
        \begin{itemize}
            \item 누설 전력 감소
            \item 동적 전력이 특정 core에만 집중될 수 있음 $\Rightarrow$ Active FU 교대시켜 해결
        \end{itemize}
    \end{itemize}
    \item For superscalar processors (Hsu and Kremer [2003])
    \begin{itemize}
        \item DVFS table을 compiler가 가지고 있음 $\Rightarrow$ DVFS 기반 DTM을 compiler를 이용하여 최적화
        \item 성능에 예민하지 않은(insensitive) 부분을 찾음 $\Rightarrow$ 해당 부분에서는 성능을 저하시킴
    \end{itemize}
    \item Many-core NoCs (Network-on-Chips) (Narayanan et al. [2007])
    \begin{enumerate}
        \item 가능한 thread는 프로세서에 mapping $\Rightarrow$ 전력 밀도 분산
        \item 전력 밀도가 높은 thread를 여러 개의 전력 밀도가 낮은 thread로 나눔
    \end{enumerate}
\end{itemize}

\subsubsection*{Liquid Cooling Techniques}
\begin{itemize}
    \item Liquid Cooling(수냉)의 필요성
    \begin{itemize}
        \item Air cooling을 이용한 냉각은 비효율적
        \item 액체는 공기보다 열용량이 큼 $\Rightarrow$ 공기보다 더 효율적인 냉각 가능
        \item 앞으로의 thermal mgmt를 위한 신흥 기술
        \item Chip의 바깥쪽에서 이루어지는 냉각 기술
    \end{itemize}
    \item Indirect liquid cooling technique (Koo et al. [2005])
    \begin{itemize}
        \item Die 사이로 구멍(micro-channel)을 뚫어 chip을 냉각시킴. 냉각제와 chip이 직접 맞닿지 않음
    \end{itemize}
    \item Direct liquid cooling technique (Brunschwiler et al. [2008])
    \begin{itemize}
        \item 3D 미세프로세서 대상 기술
        \item 냉각제가 layer 사이로 들어가 절연체를 둘러쌈, chip과 TSV(Through Silicon Vias) 사이로 흐름
        \item Outlet 수온: Data center $\rightarrow$ Liquid pumping
        \item Inlet 수온: Liquid pumping $\rightarrow$ Chip
    \end{itemize}
    \item 냉각제를 가져오는 과정에서 전력 소모가 더 심할 수 있음
\end{itemize}

\subsection{Thermal Reliability and Security}

\subsubsection*{Thermal Reliability and Security}
\begin{itemize}
    \item Performance / Power: 프로세서의 \textbf{Efficiency}
    \item Temperature: 프로세서의 \textbf{Availability}
    \begin{itemize}
        \item 프로세서의 수명 단축, 물리적 손상, System crash 등의 malfunction 등
    \end{itemize}
    \item Dadvar and Skadron [2005]: Thermal 보안 취약점으로 인해 악성코드 발생 등의 문제 발생 가능성 경고
\end{itemize}

\subsubsection*{Lifetime Reliability Model (Srinivasan et al. [2004, 05])}
\begin{itemize}
    \item MTTF (Mean time to failure): 평균 에러 발생 시간
    \item $\mathrm{MTTF} \varpropto \exp \frac{1}{T}$: 온도가 증가하면 에러 발생 확률은 급격하게 증가
    \item 프로세서의 reliability는 실행중인 application과도 관련이 있음
\end{itemize}

\subsubsection*{Dynamic Reliability Management (DRM) (Srinivasan et al. [2004, 05])}
\begin{itemize}
    \item The architectual adaptation technique by target FIT(Failure in time) value
    \item 신뢰도와 관련된 margin이 존재 $\Rightarrow$ 더 높은 성능으로 실행
    \item 그렇지 않으면 target reliability design point를 달성하기 위해 성능 저하
\end{itemize}

\subsubsection*{Aging-aware scheduling technique (Tiwari and Torrellas [2008])}
\begin{itemize}
    \item Chip은 누적 사용 시간이 길어질수록 성능이 저하 $\Rightarrow$ Chip의 수명
    \item 모든 core의 clock $f$는 가장 느린 core에 의해 결정 $\Rightarrow$ Hot task는 빠른 core에, cool task는 느린 core에 할당
    \item 느린 core의 pipeline slack margin이 빠른 core보다 더 tight함
    \item 수명 단축 현상을 분산시켜 모든 core가 동일하게 수명이 단축되도록 함
\end{itemize}

\subsubsection*{Automated stressmark generation (Joshi et al. [2008])}
\begin{itemize}
    \item 온도 센서가 적절히 배치되었는지, thermal guard-band가 적절히 설정되었는지 확인하기 위한 방법
    \item 의도적으로 특정 부분을 집중적으로 뜨겁게 만듦 $\Rightarrow$ 두 FU 사이의 온도차가 매우 크게 만듦
    \item 온도 센서가 적절히 배치되지 못하면 chip의 burn이 감지되지 못함을 이용하여 확인
\end{itemize}

\subsubsection*{Power-density induced DOS attack (Hasan et al. [2005])}
\begin{itemize}
    \item Malicious thread의 반복적 작업 $\Rightarrow$ 온도 증가 $\Rightarrow$ DTM 발생 $\Rightarrow$ 정당한 thread의 성능 저하를 유발
    \item 여러 thread를 동시에 실행시키는 SMT processor 등 전형적인 hotspot이 주요 타겟
    \item Selective sedation: Thread가 전형적인 hotspot에 접근하는 횟수를 세어 malicious 여부 판단
\end{itemize}

\subsubsection*{Thermal attack in instruction caches (Kong et al.)}
\begin{itemize}
    \item 주로 전형적인 hotspot에 대해 thermal attack에 대비 $\Rightarrow$ Instruction cache 등 대비되지 않은 FU 공격
    \item Attack on L1 instruction cache
    \begin{itemize}
        \item L1 cache의 특정 부분만 자주 접근, 다른 FU의 DTM 발생을 억제
        \item 예: Cache miss, branch misprediction등이 일어나지 않는 무한 loop 실행하는 방법
    \end{itemize}
\end{itemize}
