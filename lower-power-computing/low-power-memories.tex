\section{Low-Power Memories}

\subsection{SRAM Low-Power Techniques}

\subsubsection*{SRAM (Static RAM)}
\begin{itemize}
    \item 속도가 빠르고 밀도가 높음(DRAM보단 낮음) $\Rightarrow$ cache로 이용됨
    \item SRAM의 장점
    \begin{itemize}
        \item Memory cell의 refresh가 불필요
        \item Differential pair of bit-lines $\Rightarrow$ 빠름
        \item Row와 column의 address signal이 동시에 load $\Rightarrow$ Operation mode가 더 간단
        \item 데이터 유지(data retention) 전류가 낮음
    \end{itemize}
    \item SRAM의 단점: 면적당 용량이 작음, High cost
    \item Pins
    \begin{itemize}
        \item $\mathrm{R/\bar{W}}$: 1 = read, not write / 0 = write, not read
        \item $\mathrm{\bar{CS}}$: 1 = Chip select / 0 = Chip not select
    \end{itemize}
    \figcmd{lower-power-computing/images/sram-read-and-write.png}{.5}
    \item Active(Dynamic) Power and Standby(Static) Power Reduction
    \begin{itemize}
        \item Active Power 발생 원인: Decoders, Memory 배열, Sense amplifiers, Input/output buffer 등
        \item Standby Power 발생 원인: Data retention에 필요한 전류 (용량에 비례)
    \end{itemize}
\end{itemize}

\subsubsection*{Active Power Reduction Techniques}
\begin{itemize}
    \item $V_{\mathrm{dd}}$를 줄임 $\Rightarrow$ 동적 전력 감소
    \item Power gating, Clock gating, DVFS 등
    \item Multi-Stage Row/Column Decoder
    \begin{itemize}
        \item Heirachical Word-Line (HWL)
        \item 상위 몇개 비트를 먼저 비교 (Predecoding) 
        \item 상위 비트에 따라 서로 다른 decoder 사용하여 비교 (Final Decoding)
        \item Total capacitance 감소 $\Rightarrow$ 동적 전력 감소
        \item 속도 역시 상승 $\Rightarrow$ 많이 쓰임
    \end{itemize}
    \item Pulse Operation Technique: Address Transition Detector (ATD) Circuit
    \begin{itemize}
        \item 입력값이 바뀌었을 때만 동작하게 하는 회로
        \item 예: $a_n$의 값이 0 $\rightarrow$ 1로 바뀌는 경우
        \item XOR gate 결과: $a_1$, $a_2$, \ldots, $a_{n-1}$: 0, $a_n$에서는 delay로 인해 1이 되는 순간이 있음
        \item $\phi_{\mathrm{AID}}$ 값이 순간적으로 1이 되고, enable signal로 작동하여 동적 전력 소모 감소
    \end{itemize}
    \figcmd{lower-power-computing/images/atd-circuit.png}{.2}
\end{itemize}

\subsubsection*{Standby Power Reduction Techniques}
\begin{itemize}
    \item Data Retention
    \begin{itemize}
        \item DVFS. Drowsy Cache(정해진 전압 공급), Super Drowsy Cache(전압을 생성하여 공급)
        \item 장점: Low-power mode에서도 데이터 유지, Power Mode간 빠른 switch, 쉬운 구현
        \item 단점: Process variation에 영향 받음
        \item $P_{\mathrm{leak}} \approx$ 6.24nW
    \end{itemize}
    \item No Data Retention
    \begin{itemize}
        \item Gated $V_{\mathrm{dd}}$
        \item 장점: 전력 절약량 제일 큼, Power Mode간 빠른 switch, 쉬운 구현
        \item 단점: Low-power mode에서 데이터 유실
        \item $P_{\mathrm{leak}} \approx$ 0.02nW
    \end{itemize}
    \item $\log P_{\mathrm{leak}} \varpropto T$ $\Rightarrow$ 온도가 증가하면 누설 전력이 급격히 증가
    \item 공정미세화가 진행되면서 누설 전력의 비중이 높아지고 있음
\end{itemize}
