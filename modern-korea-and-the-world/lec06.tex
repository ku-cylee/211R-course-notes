\section{대한제국의 시대, 제국주의의 시대}

\subsection{대한제국 $\cdot$ 독립협회와 동아시아}

\subsubsection*{대한제국}
\begin{itemize}
    \item 아관파천 ('96. 2) $\Rightarrow$ 덕수궁으로 환궁 ('97. 2) $\Rightarrow$ 대한제국 건국 및 황제 즉위식 ('97. 10)
    \item 국내 개혁이 적극적으로 추진되던 시기
    \item 개혁의 이면에 존재한 그림자 역시 짙었던 시기
\end{itemize}

\subsubsection*{대한제국 시기의 개혁}
\begin{itemize}
    \item 정부
    \begin{itemize}
        \item 대한제국 황제 즉위식: 원구단과 황궁우
        \item 덕수궁 중건 $\Rightarrow$ 석조전 설계 및 건립
    \end{itemize}
    \item 민간: 독립 협회
    \begin{itemize}
        \item 독립신문: 최초의 한글신문, 영자판 동시 발행 (독립협회) $\Rightarrow$ 처음으로 대중에게 전달되는 한글전용 매체 등장
        \item 만민공동회 개최 ('98): 오늘날 종로2가 근처, 나라의 장래 문제에 대해 토론, 새로운 정책 방향 제시
    \end{itemize}
    \item 민관 합동
    \begin{itemize}
        \item 중국과의 사대관계 폐지: 청 사신을 맞이하던 영은문 철거, 독립문 건립 ('97)
    \end{itemize}
    \item 서울 시내에 전기 공급, 전차 운행
    \item 외형적으로 근대적인 문물 나타남 $\Rightarrow$ 개혁의 시대, 문명의 시대 출범 지표
\end{itemize}

\subsubsection*{러$\cdot$일간 한반도를 둘러싼 협상}
\begin{itemize}
    \item 대한제국을 둘러싼 열강들의 협상과 다툼 $\Rightarrow$ 러$\cdot$일이 한반도에 직접 관여
    \item 베베르-고무라 각서 ('96. 5. 14):
    \begin{itemize}
        \item 베베르: 주한 러시아 공사 / 고무라 주타로: 주한 일본 공사
        \item 1항: 국왕의 환궁 문제: 안전상 의구심을 품을 필요가 없다고 인정될 때 환궁 권고
        \item 2항: 일본은 현행 친러 내각을 인정
        \item 3항: 경부간 일본전신선 보호를 위한 일본 병사 200여명 주둔 허용
        \item 4항: 경성 및 개항장의 일본인 거류지 보호를 위한 일본군 주둔 허용, 러시아도 동수의 군인 주둔 허용
    \end{itemize}
    \item 로바노프-야마가타 의정서 ('96. 6. 9)
    \begin{itemize}
    \item 로바노프: 러시아 외상 / 야마가타: 일본 특사
    \item 1항: 조선이 개혁을 위한 외채를 모집 $\Rightarrow$ 양국 정부는 합의 하에 조선에 대한 원조 제공
    \item 2항: 조선의 군대 및 경찰 창설을 조선에 일임, 외부 재원 제공 불허
    \item 3항: 경부간 일본전신선 보호 유지, 러시아는 서울 \textasciitilde{} 러시아 국경 간 전신선 가설 권리 획득
    \item 비밀1항: 조선의 질서가 어지러워짐 $\Rightarrow$ 양국이 합의 하에 군을 파견, 양국군 사이에 비무장지대를 둠
    \begin{itemize}
        \item 한반도 분할론
    \end{itemize}
    \item 비밀2항: 베베르-고무라 각서의 4항을 조선이 자국군을 창설할 때까지 유지
    \end{itemize}
    \item 로젠-니시 의정서 ('98. 4. 25)
    \begin{itemize}
        \item 3항 (핵심): 러시아가 일본의 한반도에서의 상$\cdot$공업상 이익을 인정
        \begin{itemize}
            \item 당시 러시아의 이익이 만주에 집중 $\Rightarrow$ 한반도 내에서는 일본의 이익을 인정하는 쪽으로 타협
        \end{itemize}
    \end{itemize}
\end{itemize}
