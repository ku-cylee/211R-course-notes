\section{조선은 세계와 언제, 어떻게 만났는가}

\subsection{19세기 후반 서구문명과 동아시아 세계의 충돌}

\subsubsection*{서세동점: 서양의 세계/동아시아로의 진출}
\begin{itemize}
    \item 초기의 서세동점

    \begin{itemize}
        \item 15말 포르투갈이 희망봉 발견
        \item 16초 에스파냐의 마젤란의 세계 일주
        \item 16중 네덜란드의 나가사키 교역
    \end{itemize}

    \item 서세동점 시기 동방 진출의 두 주체
    \begin{itemize}
        \item 상인
        \item 선교사: 프란치스코 하비에르 신부, 마테오 리치 신부 등
    \end{itemize}
\end{itemize}

\subsubsection*{하멜의 조선 표류}
\begin{itemize}
    \item 하멜의 조선 표류
    \begin{itemize}
        \item 1567년 네덜란드가 일본 나가사키에 상관(商館)을 설치하여 일상적 교역 시행
        \item 하멜 역시 네덜란드 상인 중 한명이었음
        \item 1653년 제주 표류 $\rightarrow$ 1666년 일본으로 탈출 및 귀국 $\rightarrow$ 1688년 하멜표류기 출판
        \begin{itemize}
            \item 세 가지 판본이 있음
            \item 17세기 조선의 생활상을 세세하게 기록한 최초의 유럽 서적
            \item 여러 동물이 있고, 코끼리는 없으며, 악어는 강에 산다는 등의 내용 포함
        \end{itemize}
        \item 조선과 서양의 직접적인 첫 만남
    \end{itemize}
\end{itemize}

\subsubsection*{서세동점의 변화}
\begin{itemize}
    \item 18-19세기가 되며 서세동점의 성격이 크게 달라짐.
    \item 국가가 중심, 주체가 된 \textbf{통상 요구}
    \begin{itemize}
        \item 18세기 말 산업혁명 $\rightarrow$ 생산량 급증 $\rightarrow$ 해외시장, 세계무역의 중요성 급증
        \item 19세기 중 프/독/러/미 등 후발자본주의 국가들이 해외시장 진출 시도
        \item 아프리카, 인도, 아메리카 등은 레드오션 $\rightarrow$ 블루오션인 동아시아에 관심
    \end{itemize}
    \item 서양 세력은 동아시아 국가에 기존의 화이질서를 버리고 \textbf{국제법 질서} 수용 요구
    \begin{itemize}
        \item 화이질서 (華夷秩序): 중국 중심의 국제 질서. 중화, 조공, 책봉으로 이루어진 관계
        \item 국제법 질서: 국가간 조약의 체결을 통하여 관계를 규율
    \end{itemize}
\end{itemize}

\subsubsection*{동아시아와 서양 세력의 충돌}
\begin{itemize}
    \item 중국
    \begin{itemize}
        \item 양차 아편전쟁 (1839-42, 56-60)
        \item 아편전쟁으로 인해 수많은 불평등 조약을 수용하게 됨
        \item 남경조약(영1842), 망하조약(미1844), 황포조약(프1844), 이리조약(러1851)
        \item 북경조약(영프러1860): 조, 중, 러가 국경을 접하게 됨
        \item 양무운동(洋務運動), 중체서용(中體西用): 몸은 중국, 기술은 서양의 신문물
    \end{itemize}
    \newpage
    \item 일본
        \begin{itemize}
            \item 페리제독의 함대(흑선): 1853년 1차 입항 $\rightarrow$ 개항 요구 $\rightarrow$ 1854년 2차 입항
            \item 군함들의 배수량이 중/일 선박 최대 배수량의 25배 이상
            \item 미일화친조약 체결 (1854. 3. 31)
            \item 미일수호통상조약 체결 (1858. 7. 29): 표류민 구조 조항 등
            \item 메이지유신(明治維新), 화혼양재(和魂洋才): 정신은 야마토 정신, 기술은 서양의 신문물
        \end{itemize}
    \item 불평등 조약: 협정관세, 영사재판권, 최혜국대우 조항, 개항장 군함 정박권 등의 내용이 포함된 조약
    \item 포함외교: 외교협상에서 군사력에 의한 위협을 통해 압력을 가하여 협상을 유리하게 진행하는 외교정책
    \item 중국과 일본은 개항 이후 서양의 무력 우위를 인정하고 국내 개혁을 추진
    \item 조선
    \begin{itemize}
        \item 영국: 제주도, 울릉도(1787), 영흥만 용당포(1797), 서남해안 탐사(1816), 홍성에서의 통상요구 (1832 - 아편전쟁 10년 전)
        \item 미국: 포경선이 동래부에 표류 (1852)
        \item 러시아: 동해, 남해 탐사 (1853)
        \item 직접적인 무력충돌: 병인양요 (프1866), 신미양요 (미1871)
    \end{itemize}
\end{itemize}

\subsection{대원군 집권기와 조선의 세계}

\subsubsection*{대원군 집권기 국내 및 국제 정치}
\begin{itemize}
    \item 국내 정치
    \begin{itemize}
        \item 고종 왕위 등극 $\rightarrow$ 섭정으로 집권
        \item 비변제 폐지, 호포제(조세부담 완화) (1864)
        \item 서원 철폐령(1865), 병인사옥(1866)
        \item 서학에 반대하여 동학 창도 (1860) $\rightarrow$ 급속도로 성장 $\rightarrow$ 동학 교주 최제우 처형 (1864)
    \end{itemize}
    \item 대외 정치
    \begin{itemize}
        \item 병인박해 $\rightarrow$ 빌미삼아 프랑스가 침공 $\rightarrow$ \textbf{병인양요 (1866. 9 \textasciitilde{} 11)}
        \item 독일 상인 오페르트의 통상요구 거절 $\rightarrow$ 남연군묘 도굴 미수 사건 (1868)
        \begin{itemize}
            \item 대원군 아버지 남연군의 묘를 볼모로 통상 요구 계획
            \item 서양에 대한 조선인의 인식이 더 나빠짐
        \end{itemize}
        \item 제너럴 셔먼 호 사건 (1866. 8 \textasciitilde{} 9) $\rightarrow$ 뒷처리를 문제삼아 미국이 침공 $\rightarrow$ \textbf{신미양요 (1871)}
    \end{itemize}
    \item 평가
    \begin{itemize}
        \item 대원군 집권 10년간 국내 정책과 대외 상황은 긴밀한 연관성
        \item 왕권 강화, 조세 경감 등 정책 $\rightarrow$ 국내 반발 감안하여 대외개방을 막음
    \end{itemize}
\end{itemize}

\subsubsection*{서원 철폐령(1865), 병인사옥(1866)}

\begin{itemize}
    \item 서원은 지방 양반들의 세력 근거지였음
    \item 전국 천여개 중 43개를 제외하고 모두 단시간 내에 철폐
    \item 지방 양반 유생들의 불만 무마 필요성 $\rightarrow$ 척사론, 서학 배척 $\rightarrow$ 병인사옥
\end{itemize}
\newpage

\subsubsection*{병인양요(丙寅洋擾; 1866)}

\begin{itemize}
    \item 배경
    \begin{itemize}
        \item 병인박해 (6월): 조선의 천주교 신부 및 천주교도 8천여명 처형
    \end{itemize}
    \item 진행
    \begin{itemize}
        \item 리델 신부가 프랑스 로즈 제독에게 천주교 탄압 실상 전달 $\rightarrow$ 청에 조선 공격 계획 통보
        \item 로즈제독, 지푸에서 조선으로 출동 $\rightarrow$ 양화진 도착 후 정찰 후 퇴각
        \item 강화성 공격, 점령 (10/16) $\rightarrow$ 문수산 전투 (10/26) $\rightarrow$ 정족산성 전투 (11/9) $\rightarrow$ 퇴각 (11/21)
    \end{itemize}
    \item 결과
    \begin{itemize}
        \item 강화보의 보물 및 외규장곽 의궤 탈취
    \end{itemize}
\end{itemize}

\subsubsection*{신미양요(辛未洋擾; 1871)}
\begin{itemize}
    \item 배경
    \begin{itemize}
        \item 제너럴 셔먼호 사건(1866. 8)
        \begin{itemize}
            \item 조선의 대동강 진입 불허에도 불구하고 진입, 군민 12명 살해
            \item 화공으로 제너럴 셔먼호 격침 (9/5)
        \end{itemize}
        \item 미국, 사고 문제 뒷처리를 위한 교섭에 나섬
    \end{itemize}
    \item 진행
    \begin{itemize}
        \item 인천 앞바다 제부도 도착(5/26) $\rightarrow$ 뒷처리를 위한 교섭 진행
        \item 강화해협 측량(6/1) $\rightarrow$ 손돌목에서 미국 함대 포격
        \item 강화도 공격 $\rightarrow$ 초지진, 광성진, 덕진진 점령 (6/10) $\rightarrow$ 자진 철수 (6/12)
    \end{itemize}
    \item 결과
    \begin{itemize}
        \item 양국의 총기/함포의 수준 차이가 심했음 $\rightarrow$ 조선군의 참패
        \item 수자기가 탈취됨
        \item 남북전쟁 이후 최대의 해외원정 전쟁 $\rightarrow$ 원정 준비를 철저히 하였음
    \end{itemize}
\end{itemize}

\subsubsection*{척화비}

\begin{itemize}
    \item 양대양요 이후 대원군의 지시로 건립
    \item 지금까지 남아있는 전통 시대 비석 중 가장 많이 남아있음
    \item 내용
    \begin{itemize}
        \item 서양 오랑캐와 절대 화친하지 않으며, 매국 행위로 규정. 이를 자손만대에 경고함
        \item 병인년에 비문을 짓고 신미년에 세움
    \end{itemize}
    \item 일본의 사무라이 계급이 전투를 하지도 않고 개방한 것과 달리, 조선의 양반 계급은 막대한 피해를 입고도 개방을 거부함
\end{itemize}
