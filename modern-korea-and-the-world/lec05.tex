\section{1894년 조선과 청일전쟁, 그리고 동아시아}

\subsection{동학농민전쟁 $\cdot$ 청일전쟁의 배경과 경과}

\subsubsection*{동학농민전쟁 $\cdot$ 청일전쟁}
\begin{itemize}
    \item 진행: 1894.7 \textasciitilde{} 1895.4
    \item 전봉준 등이 조선 정부에 반(反)하여 동학농민전쟁 일으킴
    \item 동학농민군의 전주성 점령을 계기로 청일이 조선의 지배권을 두고 벌인 전쟁
    \item 근대적 무기가 동원된 한반도 최초의 전쟁
\end{itemize}

\subsubsection*{동학농민전쟁 \textasciitilde{} 청일전쟁 직전 진행}
\begin{itemize}
    \item 동학 농민운동 발생 $\rightarrow$ 황토현 전투, 전주성 점령('94. 5)
    \item 조선, 중국에 파병 요청 $\rightarrow$ 중, 천진 조약에 따라 일본에 파병사실 통보 $\rightarrow$ 일, 파병 결의
    \item 중국군 아산만 상륙 $\rightarrow$ 일본군 인천만 상륙 $\rightarrow$ 전주화약 (6. 11)
    \item 일, 조선내정개혁안을 청에 제안 $\rightarrow$ 청의 거부 $\rightarrow$ 일, 경복궁 점령 (7. 23) $\rightarrow$ 친일정권 수립
    \item 일, 청에 선전포고 (8.1) $\rightarrow$ 잠정합동조관, 대조선대일본양국동맹조약 체결
\end{itemize}

\subsubsection*{잠정합동조관}
\begin{itemize}
    \item 전문: 양국군이 서울에서 우연히 충돌한 사건(경복궁 점령)을 타협, 조정 $\rightarrow$ 조선의 독립, 자주 보장 및 통상무역 장려
    \item 1조: 조선 정부 내정 개혁을 일본이 권고한 대로 힘써 시행
    \item 2조: 경부, 경인간 철도 노선 부설
    \item 3조: 일본 정부가 경부, 경인간 설치한 군용 전화선 현상유지
    \item 4조: 전라도 연해 지방에 무역항 한 개 개방
    \item 5조: 경복궁 점령 사건을 ``우연한 충돌''로 규정, 문제삼지 않음
    \item 일본의 전쟁 수행과 관련된 최우선 요소를 반영하고 있는 조약
\end{itemize}

\subsubsection*{대조선대일본양국동맹조약}
\begin{itemize}
    \item 전문: 청군을 조선에서 철수시키는 문제를 조선이 돕도록 함
    \item 1조: 청군을 조선의 국경 밖으로 철수 $\rightarrow$ 조선의 자주 독립을 보장
    \item 2조: 일본군의 군량 마련 등의 준비 및 편의를 조선이 제공
    \item 3조: 청일전쟁이 끝나는 날 폐기
    \item 전쟁은 일본이, 지원은 조선이
\end{itemize}

\subsubsection*{청일전쟁의 진행}
\begin{itemize}
    \item 중국군 아산만 상륙 (6. 7) $\rightarrow$ 일본군 인천만 상륙 (6. 9)
    \item 성환전투 (7. 29): 충남 천안 인근
    \item 평양전투 (9. 15): 일본 승리 $\rightarrow$ 전세가 기욺
    \item 이후 전쟁은 만주에서 진행 $\rightarrow$ '95. 4 종전
\end{itemize}

\subsubsection*{동학농민군의 진압 - 진압훈령 실시}
\begin{itemize}
    \item 충주, 괴산, 청주에 군집 / 전라, 충청에 흩어져 있어 토벌할 것을 명시
    \item 후비대대(후방부대) 동원 $\rightarrow$ 연령이 높았고 군기가 해이 $\rightarrow$ 여러 문제 일으킴
    \item 동학군의 지도자는 일본이 처리, 조선군의 지휘를 일본군이 하도록 명시 $\rightarrow$ 수행은 조선군, 지휘는 일본군
    \item 3로로 나누어 진격
    \begin{itemize}
        \item 서로(西路): 수원, 천안 $\rightarrow$ 공주 $\rightarrow$ 전주 $\rightarrow$ 남원 $\rightarrow$ 나주
        \item 중로(中路): 용인 $\rightarrow$ 청주 $\rightarrow$ 성주
        \item 동로(東路): 충주 $\rightarrow$ 문경 $\rightarrow$ 대구
    \end{itemize}
    \item 동로를 먼저 장악하고 서남쪽으로 몰아서 토벌할 것을 명시
    \begin{itemize}
        \item 러시아 국경으로 도주하여 러시아 개입 명분을 주지 않기 위한 조치
    \end{itemize}
    \item 이 외의 전쟁 수행과 관련된 항목
    \item 결과: 전봉준의 일본 공사관으로의 압송, 김개남 처형 등
\end{itemize}
