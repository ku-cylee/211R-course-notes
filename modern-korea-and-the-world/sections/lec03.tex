\section{1882년 조선, 세계와 만나다}

\subsection{조선 정부의 개혁 $\cdot$ 개방정책 추진과 대응}

\subsubsection*{서양 열강과의 조약 체결}
\begin{itemize}
    \item 중, 일은 첫 근대적 조약 체결 이후 지체없이 다른 열강과도 조약 체결
    \item 조선은 강화도 조약 체결 이후 서양과 첫 조약 체결(미국)까지 6년 소요
    \item 이후 영, 프, 독, 러와 조약 체결 완료까지 6년 소요
    \item 주제: 서양과의 조약 체결이 왜 이렇게 오래 걸렸는가?
\end{itemize}

\subsubsection*{초기개화정책 ('80초)}
\begin{itemize}
    \item 강화도 조약 - 조미수호통상조약까지 두 흐름
    \item 초기개화 조선 정부의 적극적인 \textbf{내부 개혁}과 \textbf{개방 정책}의 추진
    \item 강화도 조약 이후, 새로운 정책의 추진, 입안, 실천은 '80 초부터
\end{itemize}

\subsubsection*{수신사}
\begin{itemize}
    \item 1차 수신사 ('76)
    \begin{itemize}
        \item 김기수 대표, 단시간에 돌아옴
    \end{itemize}
    \item 2차 수신사 ('80. 7)
    \begin{itemize}
        \item 김홍집 대표
        \item 무관세조약 개정, 미곡 수출 제한 등 해결 시도 $\rightarrow$ 도쿄에서 1개월간 일본과 교섭
        \item 조선책략을 가지고 귀국
    \end{itemize}
\end{itemize}

\subsubsection*{통리기무아문 설치 ('80. 12)}
\begin{itemize}
    \item 중앙정부 제도 개편 (12개 부처)
    \item 서양과 외교 대비 기구: 사대사, 교린사, 어학사, 통상사 등
    \item 군비 개혁 및 정비 기구: 군무사, 변정사, 군물사, 선함사 등
    \item 개방(開放)과 강병(强兵)에 중점
\end{itemize}

\subsubsection*{조사시찰단(일) 파견 ('81. 4)}
\begin{itemize}
    \item 일본 문물 시찰 목적
    \item 민간의 극렬한 반대 $\rightarrow$ 동래부 암행어사 가장하여 비밀리 출발
    \item '84. 4. 10 출발 $\rightarrow$ 3개월간 일본에 머묾
    \item 어윤중을 비롯하여 12인의 조사, 수원, 통사 등 62명 규모
    \item 정부기관과 산업 시설 시찰
    \item 유길준, 윤치호: 최초의 유학생으로 일본에 머묾
    \item 시찰 결과는 개화정책의 밑거름
\end{itemize}
\newpage

\subsubsection*{영선사(청) 파견 ('81, 9)}
\begin{itemize}
    \item 중국의 병기 제조학습이 목적
    \item 김유식 대표, 학도 $\cdot$ 공장 38명 등으로 구성
    \item '84. 9. 26 육로로 출발 $\rightarrow$ 임오군란 발발로 조기 귀국
    \item 톈진기기창, 수사학당 등에서 무기제조학습
    \item 기기창(機器廠) 창설, 무기제조 착수
\end{itemize}

\subsubsection*{신식군대 별기군 창설 ('81)}
\begin{itemize}
    \item 종래의 중앙군 5군영 $\rightarrow$ 무위영/장어영 2영으로 축소
    \item 일본 공사관 호리모토 레이조를 군사 문으로 초빙 $\rightarrow$ 일본식 군사훈련 실시
    \item 신식군대 전폭 지원 $\rightarrow$ 구식군인 불만 야기 $\rightarrow$ 임오군란 원인
\end{itemize}

\subsubsection*{조선책략 (朝鮮策略)}
\begin{itemize}
    \item 주일청국외교관 황준헌이 집필
    \item \textbf{친중국(親中國), 결일본(結日本), 연미국(聯美國) $\rightarrow$ 방아론(防俄論)}
    \begin{itemize}
        \item 미, 중, 일과 친하게 지내며 러시아의 남하 방어
    \end{itemize}
    \item 조선 정부는 조선책략 수용 $\rightarrow$ 국내 유포
    \begin{itemize}
        \item 카자흐스탄 남서부 이리(伊犁; Kuldja) 지역에서의 중러간 국경 분쟁 발생
        \item 러시아가 10년간 무력으로 점령 ('71 - '81)
        \item 러시아와 인도(영국), 중국 간의 완충지대 $\rightarrow$ 러시아의 점령으로 분쟁 발생
        \item 조선 정부 내에서는 러시아의 남하를 예상
        \begin{itemize}
            \item 러시아 함대가 부동항을 찾기위해 반드시 남하할 것을 예상함
            \item 6개월 전 주청 영국공사 케네디가 영국 외무성에 보고한 바와 유사
            \begin{itemize}
                \item 러시아가 베이징 진출을 위해 원산항(Port Lazareff) 점령 예상
            \end{itemize}
        \end{itemize}
    \end{itemize}
    \item 민간 $\cdot$ 재야유생의 반발
    \begin{itemize}
        \item 미국은 원래 잘 모르던 나라이며 러시아는 본디 아무런 감정이 없는 국가
        \item 가까운 나라를 배척하고 먼 나라와 가깝게 지내는 것은 타당치 않다고 주장
        \item 이만손, 강진규, 이만운 등 영남지방 유생들이 반대 상소 $\rightarrow$ 영남만인소
        \item 전국적인 운동으로 확산
    \end{itemize}
    \item 조선책략은 미국과의 조약 체결에 큰 영향을 미침
\end{itemize}

\subsection{조미수호통상조약 체결과 임오군란 (1882)}

\subsubsection*{조미수호통상조약 체결 배경 $\cdot$ 과정}
\begin{itemize}
    \item 미국
    \begin{itemize}
        \item 신미양요 후 조선에 대한 관심 일시적 후퇴
        \item '78. 3. 상하원 합동 조선 개항 결의안 가결
        \item 동아시아 함대 제독 슈펠트에게 조선과의 조약 체결 임무 부여
        \item '80. 3 일본에 중재 요청 $\rightarrow$ 일본 측 소개장 가지고 부산 방문 $\rightarrow$ 조선 거절
    \end{itemize}
    \newpage
    \item 조선
    \begin{itemize}
        \item 중국 측의 서양과 수교 권유 $\rightarrow$ 조선 측의 의향 변화 ('79)
        \item 조선책략 수용 ('80) $\rightarrow$ 정부 정책 전환
        \item 중국의 중재로 미국과 조약 체결 추진
    \end{itemize}
    \item '82. 4. 6 제물포에서 체결, '83. 4 비준
\end{itemize}

\subsubsection*{조미수호통상조약의 내용}
\begin{itemize}
    \item 1조: 거중조정(居中調整)
    \begin{itemize}
        \item 상호 국가가 제3국으로부터 불공평하고 경시당하는 일이 있으면 서로 도와줌
        \item 조선 정부는 러일전쟁 - 병합 시기 미국이 지원해줄 것이라고 착각
    \end{itemize}
    \item 5조: 관세 조항
    \begin{itemize}
        \item 수출입 관세를 명문화
        \item 일상 용품 $<10\%$, 사치품과 기호품 $<30\%$, 출항하는 토산물 $<5\%$
    \end{itemize}
    \item 7조: 아편 무역 금지
    \item 8조: 홍삼, 미곡 수출 금지
    \begin{itemize}
        \item 강화도 조약 이후 조선의 쌀값이 폭등
        \item 쌀값 안정을 위하여 조선이 요청
        \item 미국은 조선에서 미곡 수입 의향 X $\rightarrow$ 수용
    \end{itemize}
    \item 9조: 무기 수입 금지 조항
    \begin{itemize}
        \item 조선 내부 정치적 상황으로 인해 추가
    \end{itemize}
    \item 11조: 유학생 파견 교류 조항
    \begin{itemize}
        \item 유학생을 파견 시 피차간 도움 제공
    \end{itemize}
    \item 14조: \textbf{최혜국대우 조항}
    \begin{itemize}
        \item 조약에 명시되지 않은 관세는 미국이 결정
        \item 조약에서 허용되지 않은 권리, 이권, 특혜를 조선이 제3국에 허용 $\rightarrow$ 동일 조건으로 미국도 향유
    \end{itemize}
\end{itemize}

\subsubsection*{임오군란(壬午軍亂) 발생 배경 및 진행 ('82)}
\begin{itemize}
    \item 배경
    \begin{itemize}
        \item 훈련도감에서 구식 군인 해고
        \item 10개월 이상 연체된 봉급에 대해 정부가 불량쌀을 지급 $\rightarrow$ 불만 폭발
    \end{itemize}
    \item 개화추진세력 (정부) vs 개화반대세력 (대원군파)
    \item 조선이 외국과 조약을 체결한 이후 $\rightarrow$ 기존의 민란과는 전혀 다른 방향으로 전개
    \item 진행 과정
    \begin{itemize}
        \item 무위영 군인 봉기 (6. 9) $\rightarrow$ 일본 공사관 포위
        \item 일, 조선출병 결정 (6. 17) $\rightarrow$ 일본 국내 계엄령 발동 (6. 22)
        \item 청, 무력 개입 결정 (6. 25) $\rightarrow$ 인천 월미도 도착 (6. 27)
        \item 미, 모노카시호 인천 월미도 도착 (6. 30)
        \item 청군 입경(入京) $\rightarrow$ 대원군 납치, 톈진으로 호송 $\rightarrow$ 고종의 정권 회복 (7. 13)
        \item 일본과 제물포 조약, 조일수호조규속약 체결 (7. 17)
        \item 청과 조중상민수륙무역장정 체결
    \end{itemize}
    \item 이때부터 조선 문제가 중일의 국내 문제로 직결되는 계기
    \item 일본의 국내 정치 이용
    \begin{itemize}
        \item 임오군란을 메이지유신 이후 최초의 해외반일운동으로 인식
        \item 계염령 선포 $\rightarrow$ 신문 검열, 집회 통제 $\rightarrow$ 국내 자유민권운동 통제
        \item 준전시 상황 빌미로 징발령 선포 $\rightarrow$ 민간의 재산 징발
        \item 파병된 군대를 매우 허약하게 묘사 $\rightarrow$ 군비, 재정 확장에 대한 여론 조성
    \end{itemize}
\end{itemize}

\subsubsection*{임오군란의 결과}

\begin{itemize}
    \item 제물포조약 (조-일)
    \begin{itemize}
        \item 조선측에 피해보상 요구
        \item 공사관에 일본 경비병 주둔
        \item 공식 사과를 위한 수신사 파견 등
    \end{itemize}
    \item 조일수호조규속약
    \begin{itemize}
        \item 강화도 조약에서 미처 얻어내지 못한 내용 요구 $\rightarrow$ 경제분야의 내용 관철
        \item 개항장에서 일본인의 이동 거리 제약 완화
        \item 서울 양화진 개시 (開市)
        \item 조선 내 여행 허가 등
    \end{itemize}
    \item 조중상민수륙무역장정 (조-청)
    \begin{itemize}
        \item 3천명의 청군이 한성에 주둔하던 상황에서 체결
        \item 조선이 청의 ``속방''임을 강조하는 내용
        \item 양화진 개시, 중국 상인의 내륙 상행위 허용
        \item 청의 정치적, 경제적, 군사적 개입을 허용한 조약
    \end{itemize}
    \item 공사관 보호 명목으로 외국군이 주둔하게 됨
    \item 조선 내부 문제 발생 시 청, 일이 즉각적으로 군사개입 가능
    \item 조선의 국제적 분쟁지역화 $\rightarrow$ 갑신정변 시 청일간 충돌로 비화
    \item 척양비(척화비) 폐기
    \begin{itemize}
        \item 조선이 서양과 교류를 추진하고자 하는 상징적인 조처
        \item 조선의 대외개방반대 정책은 공식적으로 폐기됨
    \end{itemize}
\end{itemize}
