\section{조선, 최초의 `조약'을 체결하다}

\subsection{조선과 일본, 교린에서 조약으로}

\subsubsection*{강화도 조약 체결 배경에 대한 평가}
\begin{itemize}
    \item 내인론(內因論): 내부적 원인, 개화 사상가의 조선의 개화 필요성 주장 등
    \item 외인론(外因論): 외부적 원인, 중국과 일본의 대외정책 등
\end{itemize}

\subsubsection*{조약이란?}
\begin{itemize}
    \item 국제법 주체(평등한 국가) 간
    \item 권리/의무 관계 창출이 목적
    \item `서면' 형식으로 체결
    \item 국제법에 의하여 규율되는 합의
\end{itemize}

\subsubsection*{메이지 유신 (明治維新; '68)}
\begin{itemize}
    \item 정치: 덴노가 상징권력, 쇼군이 실질권력 $\rightarrow$ 덴노가 실질권력
    \item 행정: 270여개 다이묘가 지배하던 번(番) 폐지 $\rightarrow$ 현(縣) 설치
    \item 국민 개병제(징병제), 식산흥업(殖産興業) 등 추진
    \item 경제, 군사, 외교, 행정 등 모든 분야의 제도 변화
    \item 신분제: 지배 계급인 사무라이(武士) 계급 폐지 $\rightarrow$ 내부 불만 촉발
\end{itemize}

\subsubsection*{일본의 대조선 정책과 정한론의 대두}
\begin{itemize}
    \item 1단계: 대마도를 통한 협상 (교린 질서)
    \begin{itemize}
        \item 왜관(倭館)을 통하여 진행
        \item 서계: 조일간 확립된 전통적 외교문서(서계)
        \item 일본은 중국에서만 사용해왔던 황실, 칙명, 천황 등의 용어를 사용 $\rightarrow$ 외교문서 형식의 일방적 변경 $\rightarrow$ 조선의 외교문서 거부
    \end{itemize}
    \item 2단계: 외무성을 통한 직접 협상 ('70년 이후) $\rightarrow$ 외무성 관리 직접 파견 $\rightarrow$ 조선의 서계 접수 거부
    \item 조선정책안 3안 등장
    \begin{itemize}
        \item 1안: 조선과 단교 $\rightarrow$ 사태 방임
        \item 2안: 사신 파견 $\rightarrow$ (거절시 무력 사용) $\rightarrow$ 개항과 자유왕래 보장하는 조약 체결 $\rightarrow$ 정한론
        \item 3안: 중국과 대등한 관계 구축 $\rightarrow$ 우회적으로 조선 교섭
    \end{itemize}
    \item 일본 정부 내 두 정치세력의 대립
    \begin{itemize}
        \item 외정파: 사무라이 계급의 불만 무마를 위해 정한론(征韓論) 주장
        \item 내치파: 내치가 우선임을 주장
        \item 메이지 6년 정변: 내치파 승리 $\rightarrow$ 외정파 퇴출 $\rightarrow$ 정한론 좌절
    \end{itemize}
    \item 히로츠 노부히로의 무력시위 요청
    \begin{itemize}
        \item 조선에 파견된 외무성 관리
        \item 조선에 무력 시위를 실행하여 개방 유도할 것을 주장 $\rightarrow$ 운요호 사건 발생
    \end{itemize}
\end{itemize}

\subsubsection*{조선 내부의 논의 상황}
\begin{itemize}
    \item 승정원: 우선 서계를 접수하고 논의할 것을 주장
    \item 의정부: 절대 수용 불가 주장
\end{itemize}

\subsubsection*{운요호 사건 ('75)}
\begin{itemize}
    \item 진행
    \begin{itemize}
        \item 식수를 구한다는 명분 $\rightarrow$ 허가없이 작은 배로 한강 하구에 진입
        \item 초지진 포대의 포격 $\rightarrow$ 운요호 피해 없음
        \item 운요호의 함표 사걱 $\rightarrow$ 초지진 파괴 $\rightarrow$ 영종도 상륙 $\rightarrow$ 관민 살해 후 귀환
    \end{itemize}
    \item 운요호 사건에 대한 책임 제기, 조약 체결 요구 ('76. 01) $\rightarrow$ 일본 정부의 궁극적 목표
    \item 조선 정부의 입장: 무력시위 방지, 교린의 입장에서 만나보자
    \begin{itemize}
        \item 왜양분리론(倭洋分離論): 서양과 일본은 다르다는 주장. 정부가 주장
        \item 왜양일체론(倭洋一體論): 일본이나 서양이나 똑같다. 민간, 재야유생이 주장
        \item 왜양분리론 채택, 전권대신 신헌에게 조약 체결 전권 위임 $\rightarrow$ 강화도 조약 체결
    \end{itemize}
\end{itemize}

\subsection{1876년 강화도 조약의 내용과 성격}

\subsubsection*{강화도 조약의 구성}
\begin{itemize}
    \item 조일수호조규 ('76. 2): 흔히 부르는 강화도 조약
    \item 조일수호조규부록 ('76. 8)
    \item 조일수호조규부속왕복문서 ('76. 8)
    \item 조일무역규칙 ('76. 8)
\end{itemize}

\subsubsection*{조일수호조규(朝日修好條規)}

\begin{itemize}
    \item 전문: 조약의 성격, 체결 목적 $\rightarrow$ 옛날의 좋았던 관계를 거듭해서 다시 닦고자 함.
    \item 1조: 조선은 자주국이며, 일본과 평등한 권리 보유
    \begin{itemize}
        \item 근대 조약 체결에서 주권국가간 조약에서는 당연한 전제
        \item 조선에 대한 중국의 종주권 부정 의도
    \end{itemize}
    \item 2조: 일본 정부는 수시로 사신 파견하여 예조판서 접견 $\rightarrow$ 양국의 외교관계 변화
    \begin{itemize}
        \item 기존의 교린관계: 쓰시마를 통함, 연간 사신의 수 및 머물 수 있는 시간 한정, 왜관에만 머무르도록 제한
    \end{itemize}
    \item 3조: 문서에 사용된 문자에 대한 규정 $\rightarrow$ 일본은 국문(일본어), 조선은 진문(한문) 사용
    \begin{itemize}
        \item 한문의 표의문자 $\rightarrow$ 상황에 따른 해석의 차이 발생 가능
    \end{itemize}
    \item 4조: 부산 초량항에 공관 신설, 세견선 폐지, 상인들의 편의 보장
    \begin{itemize}
        \item 초량항 ``공관''으로 표현 (원래 왜관)
        \begin{itemize}
            \item 왜관: 주변에 담을 설치하고 문을 둠, 조선의 관리가 단속, 왜관 내에서만 교역
            \item 공관: 외교관을 파견한 국가의 영토, 치외법권 적용
            \item 일본은 조선과 상의없이 일방적으로 공관으로 표시
        \end{itemize}
        \item 세견선: 기존 왜관무역의 방식 $\rightarrow$ 폐지를 일방적으로 통지
        \item 편의 보장: 개항장 내 토지 임대, 가옥 구입 가능 $\rightarrow$ 일본인 집단거류지 설치 근거
    \end{itemize}
    \newpage
    \item 5조: 경기, 충청, 전라, 경상, 함경 5도 중 통상에 편리한 항구 2개 개항
    \begin{itemize}
        \item 인천('83), 원산('79)이 개항됨
    \end{itemize}
    \item 6조: 재해 등으로 항해 어려움 $\rightarrow$ 어느 연안이든지 항구에 들어가 선박 수리 가능
    \begin{itemize}
        \item 판단의 주체는 일본의 선박 $\rightarrow$ 언제든지 일본 선박이 항구에 들어갈 수 있도록 보장
    \end{itemize}
    \item 7조: 일본 선박의 조선 연해 수시 측량권 보장 $\rightarrow$ 해도 제작 가능
    \begin{itemize}
        \item 당시 일본에서 해도를 작성할 수 있는 주체: 민간이 아닌 일본 군부
        \item 조선 연안에서 무력 충돌 발생 시 일본이 군사적으로 이용할 수 있는 자료
        \item 일본 군함이 언제든지 조선 연안을 측량할 수 있는 근거 마련
    \end{itemize}
    \item 8조: 일본국 상민을 관리하는 관원 (영사관) 설치
    \begin{itemize}
        \item 영사: 상대국과 자국민 간에 발생하는 상황 관리
    \end{itemize}
    \item 9조: 양국 관리는 무역에 간섭 불가. 범죄 발생 시 양국 관리가 해당 상민을 처벌
    \item 10조: 조선인이 범죄 $\rightarrow$ 조선인 관원이 처벌, 일본인이 범죄 $\rightarrow$ 일본인 관원이 처벌
    \begin{itemize}
        \item 일본인이 조선의 항구에서 범죄 $\rightarrow$ 일본 영사가 처벌
        \item 단시간에 큰 이익을 노리는 일본 상인이 많았음 $\rightarrow$ 조선이 큰 피해를 입음
    \end{itemize}
    \item 11조: 통상장정 신설, 세목 보완하여 관련 조약 체결 명시
    \begin{itemize}
        \item '83. 8: 조일통상장정 체결
    \end{itemize}
    \item 12조: 조약 체결 날짜로부터 효력 발생, 조약 개정에 관한 내용은 없음
\end{itemize}

\subsubsection*{강화도 조약의 성격: 불평등 조약}
\begin{itemize}
    \item 영사재판권, 조계 설정, 해도 작성, 임의무역, 유효 및 폐기조항 결여
    \item 관세자주권이 부인: '83년까지 조일간 무관세무역
    \begin{itemize}
        \item 조선은 주로 쌀을 수출, 일본은 주로 서양의 면포를 수출
        \item 조선의 국내 쌀값이 3배 이상 상승 ('76-'82)
        \item 관세가 설정되지 못하고 무방비로 이루어진 통상의 결과
    \end{itemize}
    \item 조선이 최초로 맺은 근대적 형식의 조약. 이후 조선이 맺은 조약의 원형
\end{itemize}
