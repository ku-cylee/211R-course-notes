\section{1904년 대한제국과 러일전쟁, 그리고 동아시아}

\subsection{러일전쟁과 열강의 동아시아 재편}

\subsubsection*{러일전쟁 직전 동아시아 세계}

\begin{itemize}
    \item 중국
    \begin{itemize}
        \item 열강의 중국 분할 $\Rightarrow$ 중국 내 반외세 감정 폭발 $\Rightarrow$ 의화단운동 발생
        \item 열강의 의화단운동 진압 $\Rightarrow$ 신축각국조약 ('01. 9)
        \item 러시아: 만주지역에 군대 주둔, 영향력 강화
    \end{itemize}
    \item 영, 일: 제1차 영일동맹 ('02. 1)
    \begin{itemize}
        \item 전문: 청, 한국의 독립, 영토 보전 유지 및 두 나라에서 각국의 상공업이 균등한 기회 명시
        \item 1조: 영국은 청에서, 일본은 한국에서 취하고 있는 각별한 이익 보호
        \item 열강 중 가장 후발주자였던 일본이 가장 선발주자였던 영국과 동맹 체결
        \item 군비 확장, 국가 예산 3배 증가 $\Rightarrow$ 러시아 상대로 한국에 대한 우위를 점하기 위한 준비
    \end{itemize}
\end{itemize}

\subsubsection*{러일전쟁 (1904 - 05)}

\begin{itemize}
    \item 대한제국 대외중립선포 ('04. 1. 23)
    \item 여순항의 러시아 극동함대 기습 (2. 8), 인천항의 러시아 전함 공격 (2. 9)
    \item 일본의 선전포고 (2. 10)
    \item 일본의 강요로 한일의정서 체결 (2. 23)
    \begin{itemize}
        \item 1조: 한국은 일본을 확신하고 시정 개선에 대한 충고 수용
        \item 2조: 일본은 한국 황실을 안전강령(安全康寧)하게 한다
        \item 3조: 일본은 한국의 독립과 영토 보전을 보증
        \item 4조: 외세 침략, 내란 등으로 황실의 안녕, 영토 보전에 위험 $\Rightarrow$ 일본은 조치를 취할 수 있음
        \begin{itemize}
            \item 한국은 일본에 편의 제공 의무, 일본은 정황에 따라 전략상 필요한 지점을 차지 가능
            \item 러시아의 침략, 동학농민전쟁 등의 봉기 등을 상정
        \end{itemize}
        \item 한국은 러일전쟁 중 일본의 군사동맹국으로써 행동할 수밖에 없었음
    \end{itemize}
    \item 한국주차군 설립 (3. 10) $\Rightarrow$ 한국주차군군율 제정 (7. 2)
    \begin{itemize}
        \item 1조: 군용전선 $\cdot$ 철도 파손 시 사형
        \item 2조: 파손범을 알고도 은닉하는 경우 사형
        \item 5조: 군용전선 $\cdot$ 철도의 보호는 가설된 마을 전체의 인민 책임
        \item 6조: 군용전선 $\cdot$ 철도의 파손범이 체포되지 않은 경우 당일의 보호위원을 처벌
        \item 8조: 실수로 군용전선 $\cdot$ 철도를 파손한 자는 처벌, 침구와 음식은 본인 부담
        \item 민간인에게도 적용되는 군율
    \end{itemize}
    \item 일본 내각의 대한방침, 대한시설강령 결정 (5. 31) $\Rightarrow$ 전후 한국 지배를 위한 정책 마련
    \item 제1차 한일협약 체결 (8. 22)
    \begin{itemize}
        \item 일본이 추천하는 일본인 1인을 재정고문, 외국인 1인을 외교고문으로 초빙
    \end{itemize}
    \item 여순항, 봉천전투에서 일본 승리 ('05. 1 - 2), 쓰시마해전에서 일본 승리 ('05. 5)
    \newpage
    \item 열강의 동아시아 분할 조정
    \begin{itemize}
        \item 가쓰라 - 태프트 밀약 (미-일, 7. 27): 필리핀/한국에 대한 상호 양해
        \item 제2차 영일동맹 (영-일, 8.12): 인도/한국에 대한 상호 양해
        \item 포츠머스 조약 (러-일, 9. 5): 한국에 대한 일본의 보호 승인
        \item 삼국간섭이 다시 일어나지 않게끔 사전에 교섭을 진행함
    \end{itemize}
\end{itemize}

\subsection{일본의 보호국화와 대한제국의 저항}

\subsubsection*{제 2차 한일협약 (을사늑약; '05. 11. 17)}

\begin{itemize}
    \item 일본에 의해 강제로 추진된 조약
    \item 내용
    \begin{itemize}
        \item 1조: 일본 정부는 도쿄의 외무성을 통해 한국의 대외 관계 및 사무 감독, 재외 한국 관리 및 백성 보호
        \item 2조: 일본은 한국과 타국 간의 현존하는 조약의 실행 책임, 한국은 타국과 조약 체결 시 반드시 일본의 중개를 거침
        \item 3조: 일본은 외교 사항 관리하는 통감을 황제 하에 1인을 둠, 개항장 등에는 이사관을 두어 주한일본영사의 업무를 이관
        \item 4조: 한일간 현존하는 조약은 본 협약에 저촉되는 것을 제외하고는 모두 효력이 유지
        \item 5조: 일본은 한국 황실의 안녕과 존엄 유지를 보증
    \end{itemize}
\end{itemize}

\begin{itemize}
    \item 을사5적: 을사늑약 체결에 가담한 한국 정부 대신
    \begin{itemize}
        \item 학부대신 이완용, 군부대신 이근택, 내부대신 이지용, 농상대신 권중현, 외부대신 박제순
    \end{itemize}
\end{itemize}

\subsubsection*{을사늑약에 대한 한국인의 저항}

\begin{itemize}
    \item 전민족적인 저항 촉발
    \item 조약 무효화 상소, 매국역적 처형을 주장하는 집회, 상인들의 철시(撤市) 운동
    \item 고위관료에서 무명 인력거꾼까지 10여명이 넘는 사람들의 자결 (민영환 등)
    \item 조약 체결 비판으로 300명 이상 투옥
    \item 시일야방성대곡(是日也放聲大哭)
    \begin{itemize}
        \item 장지연이 황성신문에 투고한 사설 ('05. 11. 20)
        \item 이토 히로부미는 동양3국의 분열을 야기함
        \item 고종 황제는 지속적으로 거절하였으나, 이토와 을사5적의 위협으로 결국 체결
        \item 시일야방성대곡 투고로 정간(停刊)
    \end{itemize}
\end{itemize}

\subsubsection*{보호국 체제하의 한국인의 저항}

\begin{itemize}
    \item 의병전쟁
    \begin{itemize}
        \item 을사의병 ('05): 유생 중심
        \item 정미의병 ('07): 유생, 평민 의병장, 전국적 봉기
        \item 중부지방, 남부지방에서 주로 활동이 이루어짐
        \item 남한대토벌작전: 1909년 마지막까지 남은 의병들을 진압한 작전
        \begin{itemize}
            \item 전라 지역의 의병을 서남쪽으로 몰고 포위하여 진압함
        \end{itemize}
    \end{itemize}
    \item 계몽운동, 실력양성운동, 자강운동: 언론, 학회 활동
    \newpage
    \item 신민회 ('07): 무력운동 + 계몽운동 결합형태
    \begin{itemize}
        \item 통감부의 탄압으로 인해 비밀조직 운영
        \item 미국의 안창호 중심으로 조직, 여러 인물 참여
        \item 계몽운동 시기 명망가 다수 참여 (105인 사건)
        \item 병합 이후 만주독립군 양성 $\Rightarrow$ 신흥무관학교로 연결
    \end{itemize}
    \item 매국노에 대한 직접적인 공격형태: 안중근 ('09. 10)
\end{itemize}
