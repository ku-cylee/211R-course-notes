\section{Linear Algebra Review}

\subsection{Vector Spaces}

\subsubsection*{Vector Space and Subspace}
\begin{itemize}
    \item Vector space $\mathcal{V}$ consists of
    \begin{itemize}
        \item A set of vectors
        \item Addition operator
        \item Multiplication with scalar
        \item Special element $\mathbf{0}$ vector
    \end{itemize}
    \item Subspace of a vector space: Subset of a vector space \& Itself is a vector space
\end{itemize}

\subsubsection*{Basis and Dimension}
\begin{itemize}
    \item $\mathcal{B} = \{\mathbf{v}_1,~\mathbf{v}_2,~\cdots,~\mathbf{v}_n\}$ is a basis of vector space $\mathcal{V}$ if
    \begin{itemize}
        \item $\mathcal{B}$ spans $\mathcal{V}$, i.e. $\mathcal{V} = \mathrm{span}(\mathbf{v}_1,~\mathbf{v}_2,~\cdots,~\mathbf{v}_n)$
        \item $\mathbf{v}_1,~\mathbf{v}_2,~\cdots,~\mathbf{v}_n$ are linearly independent
    \end{itemize}
    \item Dimension: $\mathrm{dim}(\mathcal{V}) = |~\mathcal{B}~|$, $\mathrm{dim}(\{\mathbf{0}\}) = 0$
\end{itemize}

\subsubsection*{Range}
\begin{itemize}
    \item $A \in \mathbb{R}^{m\times n}$, $\mathcal{R}(A) = \{A\mathbf{x}~|~\mathbf{x}\in\mathbb{R}^n\}$
    \item $A = \begin{mtx}{cccc} \mathbf{a}_1&\mathbf{a}_2&\cdots&\mathbf{a}_n \end{mtx}$,
        then $\mathcal{R} = \mathrm{span}(\mathbf{a}_1,~\mathbf{a}_2,~\cdots,\mathbf{a}_n)$
\end{itemize}

Interpretation
\begin{itemize}
    \item $\mathbf{v} \in \mathcal{R}(A)$, $\mathbf{w} \notin \mathcal{R}(A)$, $\mathbf{y} = A\mathbf{x}$: output of a sensor to input $\mathbf{x}$
    \begin{itemize}
        \item $\mathbf{y} = \mathbf{v}$: possible/consistent output, $\mathbf{y} = \mathbf{w}$: impossible/inconsistent output
    \end{itemize}
    \item $\mathcal{R}(A)$: achievable output set
\end{itemize}

\subsubsection*{Nullspace}
\begin{itemize}
    \item Nullspace of $A$: $\mathcal{N}(A) = \{\mathbf{x} \in \mathbb{R}^n~|~A\mathbf{x} = \mathbf{0}\}$
    \item Vectors of $\mathcal{N}(A)$ are orthogonal to the rows of $A$
\end{itemize}

Interpretation
\begin{itemize}
    \item $\mathcal{N}(A)$ gives ambiguity of system $A$
    \item $\forall \mathbf{x} \in \mathbb{R}^n$, $\forall \mathbf{v} \in \mathcal{N}(A)$, $A(\mathbf{x} + \mathbf{v}) = A(\mathbf{x})$
\end{itemize}

\subsubsection*{Rank}
\begin{itemize}
    \item Definition
    \begin{itemize}
        \item $A \in \mathbb{R}^{m \times n}$, $\mathrm{rank}(A) \equiv \mathrm{dim}(\mathcal{R}(A))$
        \item Number of independent columns(rows) of $A$ $\Rightarrow$ $\mathrm{rank}(A^T) = \mathrm{rank}(A)$
        \item Represents degree of freedom
    \end{itemize}
    \pagebreak
    \item Properties
    \begin{itemize}
        \item $\mathrm{rank}(A) \leq \min(m, n)$: Degree of freedom cannot be exceeded
        \item $\mathrm{rank}(BC) \leq \min(\mathrm{rank}(B), \mathrm{rank}(C))$
    \end{itemize}
\end{itemize}

\subsubsection*{Full Rank}
\begin{itemize}
    \item Definition
    \begin{itemize}
        \item $A \in \mathbb{R}^{m \times n}$ has full rank if $\mathrm{rank}(A) = \min (m, n)$
        \item $m = n$ (square matrix) $\Rightarrow$ $\exists A^{-1}$
        \item $m \geq n$ (skinny matrix) $\Rightarrow$ All columns are independent
        \item $m \leq n$ (fat matrix) $\Rightarrow$ All rows are independent
    \end{itemize}
    \item $A$ is full rank and $\mathrm{rank}(A) = m$ $(m \leq n)$
    \begin{itemize}
        \item Columns of $A$ spans $\mathbb{R}^m$
        \item Rows of $A$ are independent
        \item $A^T\mathbf{c}$ $\Rightarrow$ $\mathbf{c} = \mathbf{0}$
        \item $\det(AA^T) \neq 0$
        \item $A$ has right inverse: $\exists B$ s.t. $AB = I$ with $B = A^T(AA^T)^{-1}$
    \end{itemize}
    \item $A$ is full rank and $\mathcal{N}(A) = \{\mathbf{0}\}$
    \begin{itemize}
        \item $\mathrm{rank}(A) = n$
        \item Columns of $A$ are independent
        \item $A\mathbf{c} = \mathbf{0}$ $\Rightarrow$ $\mathbf{c} = \mathbf{0}$
        \item $\det(AA^T) \neq 0$
        \item $A$ has left inverse: $\exists B$ s.t. $BA = I$
    \end{itemize}
\end{itemize}

\subsection{Determinant}

\subsubsection*{Definition}
\begin{itemize}
    \item Function that maps a square matrix to a real number
    \item Algebraic Def: Recursively defined using $(n-1) \times (n-1)$ determinant
    \item Geometric Def: Signed value of parallelpiped formed by columns of $A$
\end{itemize}

\subsubsection*{Properties}
\begin{itemize}
    \item Multilinear:
    \begin{itemize}
        \item $\begin{detmtx}{ccccc}
            \mathbf{a}_1&\cdots&\mathbf{v}_1+\mathbf{v}_2&\cdots&\mathbf{a}_n
        \end{detmtx} = \begin{detmtx}{ccccc}
            \mathbf{a}_1&\cdots&\mathbf{v}_1&\cdots&\mathbf{a}_n
        \end{detmtx} + \begin{detmtx}{ccccc}
            \mathbf{a}_1&\cdots&\mathbf{v}_2&\cdots&\mathbf{a}_n
        \end{detmtx}$
    \end{itemize}
    \item Scaling of columns: 
    \begin{itemize}
        \item $\begin{detmtx}{cccccc}
            \mathbf{a}_1&\mathbf{a}_2&\cdots&c\mathbf{a}_i&\cdots&\mathbf{a}_n
        \end{detmtx} = c \begin{detmtx}{cccccc}
            \mathbf{a}_1&\mathbf{a}_2&\cdots&\mathbf{a}_i&\cdots&\mathbf{a}_n
        \end{detmtx}$
    \end{itemize}
    \item Determinant is zero if two columns are same: 
    \begin{itemize}
        \item$\begin{detmtx}{cccccc}
            \mathbf{a}_1&\cdots&\mathbf{a}_i&\mathbf{a}_i&\cdots&\mathbf{a}_n
        \end{detmtx} = 0$
    \end{itemize}
\end{itemize}
\pagebreak

\subsection{Eigenvalues and Symmetric Matrices}

\subsubsection*{Definition}
For $A \in \mathbb{R}^n$, $\lambda \in \mathbb{R}$ is called eigenvalue and
$\mathbf{v} \in \mathbb{R}\setminus\{\mathbf{0}\}$ is called eigenvector if $ A\mathbf{v} = \lambda\mathbf{v} $

\subsubsection*{Eigenvalues of Symmetric Matrices}
\begin{itemize}
    \item For a real matrix $A \in \mathbb{S}^n$,
    \begin{itemize}
        \item eigenvalues are real
        \item eigenvectors are mutually orthogonal
    \end{itemize}
    \item Spectral Decomposition
    \begin{itemize}
        \item $A \in \mathbb{S}^n$ can be decomposed as $A = U\Lambda U^T$, i.e. $AU = U\Lambda$
        \item $\Lambda$: Diagonal matrix with $\lambda_i$ at $i$-th diagonal,
            $U$: Orthogonal matrix $\Rightarrow$ $U^T = U^{-1}$
    \end{itemize}
    \item Positive Definiteness of Symmetric Matrices
    \begin{itemize}
        \item $A = U\Lambda U^T = \sum_i \lambda_i \mathbf{u}_i \mathbf{u}_i^T$: Sum of rank-1 matrices
        \item $\mathbf{u}_i$: bases $\Rightarrow$ $\mathbf{x} = \sum_{i} \hat{\mathbf{x}_i}\mathbf{u}_i$
            $\Rightarrow$ $A\mathbf{x} = \sum_{i} \lambda_i \hat{\mathbf{x}_i}\mathbf{u}_i$
            $\Rightarrow$ $\mathbf{x}^TA\mathbf{x} = \sum_{i} \lambda_i \hat{\mathbf{x}_i}^2$
        \item $A$: Positive definite $\Leftrightarrow$ $\forall i$, $\lambda_i \geq 0$
        \item $A$: Invertible $\Leftrightarrow$ $\forall i$, $\lambda_i \neq 0$
        \item $\lambda_i$: Eigenvalues of $A$ $\Rightarrow$ $\lambda_i^k$: Eigenvalues of $A^k$
    \end{itemize}
\end{itemize}

\subsection{Quadratic Form}

\subsubsection*{Definition}
\begin{itemize}
    \item A function that maps length $n$ vector to scalar s.t. $f(x) = \mathbf{x}^TQ\mathbf{x}$ ($Q \in \mathbb{S}^n$)
    \item $Q \notin \mathbb{S}^n$ can be replaced by $\tilde{Q} = \frac 1 2 \left(Q + Q^T \right) \in \mathbb{S}^n$,
        so assuption $Q \in \mathbb{S}^n$ does not lose generality
\end{itemize}

\subsubsection*{Positive (Semi-)Definite Matrices}
\begin{itemize}
    \item Definition
    \begin{itemize}
        \item Positive Definite (PD) Matrix: 
            $ \forall \mathbf{x} \in \mathbb{R}^n \setminus \{\mathbf{0}\},~ \mathbf{x}^TQ\mathbf{x} > 0 \Rightarrow Q \succ 0$
        \item Positive Semi-definite (PSD) Matrix: 
            $ \forall \mathbf{x} \in \mathbb{R}^n \setminus \{\mathbf{0}\},~ \mathbf{x}^TQ\mathbf{x} \geq 0 \Rightarrow Q \succeq 0$
        \item Negative Definite Matrix: 
            $ \forall \mathbf{x} \in \mathbb{R}^n \setminus \{\mathbf{0}\},~ \mathbf{x}^TQ\mathbf{x} < 0 \Rightarrow Q \prec 0$
        \item Negative Semi-definite Matrix: 
            $ \forall \mathbf{x} \in \mathbb{R}^n \setminus \{\mathbf{0}\},~ \mathbf{x}^TQ\mathbf{x} \leq 0 \Rightarrow Q \preceq 0$
        \item Indefinite Matrix: Sign of $\mathbf{x}^TQ\mathbf{x}$ differs by $\mathbf{x}$
    \end{itemize}
    \item Notations
    \begin{itemize}
        \item $\mathbb{S}^n$: Set of symmetric $n \times n$ matrices
        \item $\mathbb{S}^n_+$: Set of symmetric PSD $n \times n$ matrices
        \item $\mathbb{S}^n_{++}$: Set of symmetric PD $n \times n$ matrices
    \end{itemize}
\end{itemize}
\pagebreak

\subsubsection*{Positive Definiteness and Eigenvalues ($Q \in \mathbb{S}^n$)}
\begin{itemize}
    \item $\mathbf{x}^T Q\mathbf{x} = \mathbf{x}^T (U\Lambda U^T) \mathbf{x}
        = \mathbf{x}^T \left(\sum_i \lambda_i \mathbf{u}_i \mathbf{u}_i^T\right) \mathbf{x}
        = \sum_i c_i^2 \lambda_i$ $\left(\mathbf{x} = \sum_i \mathbf{c}_i \mathbf{u}_i\right)$
    \item $Q$ is real matrix $\Rightarrow$ Eigenvalues are all real
    \item $Q \succ 0$ $\Leftrightarrow$ $\lambda_i > 0$, $Q \succeq 0$ $\Leftrightarrow$ $\lambda_i \geq 0$
    \item $Q \prec 0$ $\Leftrightarrow$ $\lambda_i < 0$, $Q \preceq 0$ $\Leftrightarrow$ $\lambda_i \leq 0$
    \item $Q$ is indefinite $\Leftrightarrow$ There are both positive and negative $\lambda_i$s.
    \item $Q \succ 0$ $\Rightarrow$ Diagonal elements $> 0$, $Q \succeq 0$ $\Rightarrow$ Diagonal elements $\geq 0$
\end{itemize}

\subsubsection*{Properties of Positive Definiteness}
\begin{itemize}
    \item $A, B \in \mathbb{S}_{++}^n$
    \begin{itemize}
        \item $\exists A^{-1}$, $A^{-1} \succ 0$ ($A^{-1} = QD^{-1}Q^T$, Eigenvalues are $\frac 1 {\lambda_i}$)
        \item $A + B \succ 0$
    \end{itemize}
    \item $A \in \mathbb{R}^{m \times n}$
    \begin{itemize}
        \item $\mathbf{x}^T\left(A^TA\right)\mathbf{x} = (A\mathbf{x})^T(A\mathbf{x}) = \Vert A\mathbf{x} \Vert^2 \geq 0$
        \item $A^TA \succeq 0$ ($\mathrm{rank}(A) = n~\Rightarrow~A^TA \succ 0$)
        \item $AA^T \succeq 0$ ($\mathrm{rank}(A) = m~\Rightarrow~AA^T \succ 0$)
    \end{itemize}
    \item $\mathbf{v}\mathbf{v}^T \in \mathbb{S}_+^n$: Rank-1 matrix
    \item $A \in \mathbb{S}_+^n$ $\Rightarrow$ $A^k \in \mathbb{S}_+^n$
\end{itemize}

\subsubsection*{Positive Definiteness and Determinants}
\begin{itemize}
    \item Let $Q = \begin{mtx}{cc}a&b\\b&c\end{mtx}$
    \item $|Q - \lambda I| = 0$ $\Rightarrow$ $\lambda^2 - (a + c)\lambda + \left( ac - b^2 \right) = 0$
    \item Discriminant: $D = (a - c)^2 + 4b^2 \geq 0$ $\Rightarrow$ All eigenvalues are real numbers
    \item Two positive solutions: $a + c > 0$, $ac - b^2 > 0$ $\Rightarrow$ $a > 0$, $c > 0$, $ac - b^2 > 0$
    \item $Q \in S_{++}^2$ $\Rightarrow$ Diagnoal values are positive, $|Q| > 0$
\end{itemize}

\subsubsection*{Ellipse}
\begin{itemize}
    \item $\left\{\mathbf{x}~|~\mathbf{x}^TQ\mathbf{x}=c,~Q \succ 0\right\}$ is an ellipse
    \item If $Q = I$, the given set is a sphere
    \item $1 = \mathbf{x}^T Q\mathbf{x} = \sum_i \lambda_i y_i^2$ $\Rightarrow$
        $\sum_i \frac {y_i^2} {\left( \sqrt{\frac 1 {\lambda_i}} \right)^2} = 1$
    \item Axis Lengths: $\frac 2 {\sqrt {\lambda_i}}$ $\Rightarrow$ Eigenvectors with larger eigenvalue: shorter axis length
    \item Axis Directions: Eigenvectors of $Q$
\end{itemize}

\subsubsection*{Rayleigh Quotient}
\begin{itemize}
    \item Definition: $R(Q, \mathbf{x}) = \frac {\mathbf{x}^T Q\mathbf{x}} {\mathbf{x}^T\mathbf{x}}$ for $Q \in \mathbb{S}^n$
    \item $\mathbf{x}$: unit vector $\Rightarrow$ $R(Q, \mathbf{x}) = \mathbf{x}^TQ\mathbf{x}$
    \item $Q \succ 0$ $\Rightarrow$ Eigenvector with maximum(minimum) eigenvalue maximizes(minimizes) $R(Q, \mathbf{x})$
    \item Proof: $\max_{\mathbf{x} \neq \mathbf{0}} \frac {\mathbf{x}^T Q\mathbf{x}} {\mathbf{x}^T\mathbf{x}}
        = \max_{\mathbf{x} \neq \mathbf{0}} \frac {\sum_i \lambda_i \mathbf{x}_i^2} {\sum_i \mathbf{x}_i^2}
        \leq \max_{\mathbf{x} \neq \mathbf{0}} \frac {\lambda_{\mathrm{max}}\sum_i \mathbf{x}_i^2} {\sum_i \mathbf{x}_i^2}
        = \lambda_{\mathrm{max}}$
\end{itemize}
