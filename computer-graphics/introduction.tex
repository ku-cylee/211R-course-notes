\section{Introduction}

\subsection{3D Computer Graphics}

\subsubsection*{3D Computer Graphics Process}
\begin{itemize}
    \item Major steps: Modeling $\rightarrow$ Rigging $\rightarrow$ Animation $\rightarrow$ Rendering $\rightarrow$ Post-processing
    \item Modeling {\textasciitilde} Animation: 디자이너의 영역, Off-line animation
    \item Animation {\textasciitilde} Post-processing: 개발자의 영역, Run-time animation
\end{itemize}

\subsubsection*{Modeling}
\begin{itemize}
    \item Model: Computer representation of an object
    \item Modeling: Process of creating the objects comprising the virtual scenes
    \item 대부분의 실시간 3D 모델은 다각형의 집합(polygon meshes)으로 이루어짐
    \item Modeling에는 texture도 포함됨
\end{itemize}

\subsubsection*{Rigging}
\begin{itemize}
    \item Model의 skeleton(rig)를 구성
    \item Model이 움직임에 따라 skeleton이 어떻게 움직이고, model의 polygon mesh가 어떻게 움직이는지 정의
\end{itemize}

\subsubsection*{Animation}
\begin{itemize}
    \item Skeleton의 움직임을 생성하는 과정
    \item 프레임 단위로 움직임을 생성하여 화면에 보여줌
\end{itemize}

\subsubsection*{Rendering}
\begin{itemize}
    \item 3D scene에서 2D image를 생성하는 과정
    \item Lighting, texturing 등의 과정도 진행됨
\end{itemize}

\subsubsection*{Post-processing}
\begin{itemize}
    \item 렌더링된 image에 별도의 효과를 주는 optional 과정
    \item Motion blur 등이 있음
\end{itemize}

\subsection{Game Engine and Graphics API}

\subsubsection*{Game Engine}
\begin{itemize}
    \item Run-time animation인 rendering {\textasciitilde} post-processing을 수행해주는 프로그램
    \item Unity, Unreal Engine 등의 게임 엔진이 널리 사용됨
\end{itemize}

\subsubsection*{Graphics API}
\begin{itemize}
    \item 게임 엔진은 Direct3D, OpenGL과 같은 3D Graphics API를 기반으로 개발됨
    \item Graphics app에 필수적이고 GPU에 최적화된 함수들을 제공함
\end{itemize}
